\chapter{Penalized multivariate regression}
\minitoc


\section{Ridge regression}
In the case where $X^\top X$ is singular (resp. has eigenvalues close to zero), the least squares estimate cannot be computed (resp. is not robust). A common approach to control the estimator variance is to solve the surrogate Ridge regression problem
\[
\widehat \param^{\mathrm{ridge}}_n\in  \argmin_{\param\in\rset^d}  \|Y - X\param\|_2^2 + \lambda\|\param\|_2^2\eqsp,
\]
where $\lambda>0$. The matrix $X^\top X + \lambda I_n$ is definite positive for all $\lambda>0$ as for all $u\in\rset^d$,
\[
u^\top (X^\top X + \lambda I_n)u = \|Xu\|_2^2 + \lambda \|u\|_2^2\eqsp,
\]
which is positive for all $u\neq 0$. This remark allows to obtain the following result.

\begin{shaded}
\begin{proposition}
\label{prop:least:squares:ridge}
The unique solution to the Ridge regression problem is given by
\[
\widehat \param^{\mathrm{ridge}}_n = (X^\top X + \lambda I_n)^{-1}X^\top Y\eqsp.
\] 
In the well-specified setting, this estimator is biased and satisfies 
\begin{align*}
\bE[\widehat \param^{\mathrm{ridge}}_n ] - \param_*&= - \lambda(X^\top X + \lambda I_n)^{-1}\param_\star\eqsp,\\
\bV[\widehat \param^{\mathrm{ridge}}_n ] &= \sigma_\star^2(X^\top X + \lambda I_n)^{-2}X^\top X \eqsp.
\end{align*}
\end{proposition}
\end{shaded}
\begin{proof}
The unique expression of $\widehat \param^{\mathrm{ridge}}_n $ is obtained similarly as in the proof of Proposition~\ref{prop:least:squares:full:rank}. Then,
$$
\bE[\widehat \param^{\mathrm{ridge}}_n] =  (X^\top X + \lambda I_n)^{-1}X^\top \bE[Y] =  (X^\top X + \lambda I_n)^{-1}X^\top X\param_\star \eqsp.
$$
As the matrix $X^\top X$ is symmetric and real, there exists nonnegative eigenvalues $\lambda_1\geqslant \ldots \geqslant \lambda_d$ and orthonormal eigenvectors $u_1, \ldots u_d$ in $\rset^d$ such that $X^\top X = \sum_{i=1}^d \lambda_i u_i u_i^\top$ and $(X^\top X + \lambda I_n)^{-1} = \sum_{i=1}^d (\lambda_i + \lambda)^{-1}u_i u_i^\top$. Therefore, 
$$
\bE[\widehat \param^{\mathrm{ridge}}_n] -\param_\star=  \sum_{i=1}^d \lambda_i(\lambda_i + \lambda)^{-1}u_i u_i^\top\param_\star -  \sum_{i=1}^d u_i u_i^\top\param_\star =  - \lambda  \sum_{i=1}^d (\lambda_i + \lambda)^{-1}u_i u_i^\top\param_\star = - \lambda(X^\top X + \lambda I_n)^{-1}\param_\star\eqsp.
$$
Similarly,
$$
\bV[\widehat \param^{\mathrm{ridge}}_n]  =  (X^\top X + \lambda I_n)^{-1}X^\top \bV[Y]X  (X^\top X + \lambda I_n)^{-1} = \sigma_\star^{2}(X^\top X + \lambda I_n)^{-1}X^\top X  (X^\top X + \lambda I_n)^{-1}\eqsp.
$$
Therefore,
$$
\bV[\widehat \param^{\mathrm{ridge}}_n]  =   \sigma_\star^{2}(X^\top X + \lambda I_n)^{-2}X^\top X\eqsp.
$$
\end{proof}
The mean square error of the estimator is then given by
\[
\bE\left[\left\|\widehat \param^{\mathrm{ridge}}_n-\param_*\right\|_2^2\right] = \mathrm{Trace}\left(\bV[\widehat \param^{\mathrm{ridge}}_n]\right) + \left\|\bE[\widehat \param^{\mathrm{ridge}}_n] - \param_*\right\|_2^2\eqsp.
\]
Let $(\vartheta_1,\ldots,\vartheta_d)$ be an orthonormal basis of $\rset^d$ of eigenvectors of $X^\top X$ associated with the eigenvalues $(\gamma_1,\ldots,\gamma_d)\in\rset^d$. Then,
\[
\bE\left[\left\|\widehat \param^{\mathrm{ridge}}_n-\param_*\right\|_2^2\right] =  \sigma_\star^2 \sum_{j=1}^d \frac{\gamma_j}{(\gamma_j+\lambda)^2} + \lambda^2  \sum_{j=1}^d \frac{\langle \param_* \eqsp; \vartheta_j\eqsp\rangle^2}{(\gamma_j+\lambda)^2}\eqsp.
\]
The mean square error is therefore a sum of two contributions, a bias related term which increases with $\lambda$ and a variance related term which decreases with $\lambda$. In practice, the value of $\lambda$ is chosen using cross-validation.

\section{Lasso regression}
The Least Absolute Shrinkage and Selection Operator (Lasso) regression is a $\mathrm{L}_1$ based regularized regression which aims at fostering sparsity. The objective is to solve the following minimization problem,
\[
\widehat \param^{\mathrm{lasso}}_n\in  \argmin_{\param\in\rset^d}  \|Y - X\param\|_2^2 + \lambda\|\param\|_1\eqsp,
\]
where $\lambda>0$ and
\[
\|\param\|_1 = \sum_{j=1}^d|\param_j|\eqsp.
\]
The function $\param \mapsto \|Y - X\param\|_2^2 + \lambda\|\param\|_1$ is convex but not differentiable and the solution to this problem may  not be unique. For all $\param \in \rset^d$,  
\[
\partial_\param \|Y - X\param\|_2^2 = - 2 X^\top (Y-X\param)\eqsp.
\]
Then, for all $1\leqslant j \leqslant d$, $(\partial_\param \|Y - X\param\|_2^2)_j = -2 {\bf X}^\top_j (Y-X\param)$, where ${\bf X}_j$ is the $j$-th column of the matrix $X$. Define, for all $1\leqslant j \leqslant d$,
\[
\upsilon_{j}={\bf X}^\top_{j}\left(Y-\sum_{\substack{i=1\\ i\neq j}}^n\param_{i}{\bf X}_{i}\right)\eqsp,
\]
then,
\[
(\partial_\param \|Y - X\param\|_2^2)_j = -2( \upsilon_j - \param_j)\eqsp.
\]
Consequently, for all $\param_j \neq 0$, 
\[
\partial_j ( \|Y - X\param\|_2^2 +  \lambda\|\param\|_1)= 2( \param_j - \upsilon_j + \lambda\textrm{sign}(\param_j)/2)\eqsp.
\]
For all $1\leqslant j\leqslant d$,  $\param_j \mapsto  \|Y - X\param\|_2^2 + \lambda\|\param\|_1$ is convex and grows to infinity when $|\param_j|\to \infty$ and admits thus a minimum at some $\param_j^{\star}\in\rset$. 
\begin{enumerate}[-]
\item If $\param_j^{\star} \neq 0$, then
\[
\param_j^{\star} = \upsilon_j\left( 1 - \frac{\lambda ~\textrm{sign}(\param_j^{\star})}{2 \upsilon_j}\right)\eqsp,
\]
which yields, as  $\textrm{sign}(\param_j^{\star}) = \textrm{sign}(\upsilon_j)$,
\[
\param_j^{\star} = \upsilon_j\left(1 - \frac{\lambda}{2 |\upsilon_j|}\right)
\]
and
\[
1 - \frac{\lambda}{2 |\upsilon_j|} \geqslant 0\eqsp.
\]
\item If $1 - \lambda/(2 |\upsilon_j|)<0$, there is no solution to $\partial_j ( \|Y - X\param\|_2^2 +  \lambda\|\param\|_1)=0$ for $\param_j \neq 0$.  Since $\param_j \mapsto  \|Y - X\param\|_2^2 + \lambda\|\param\|_1$ admits a minimum, $\param_j^{\star}=0$. 
\end{enumerate}
Therefore,
\[
\param_j^{\star} = \upsilon_j\left( 1 - \frac{\lambda}{2 |\upsilon_j|}\right)_+\eqsp.
\]
An algorithm to approximatively solve the Lasso regression problem proceeds as follows.