%\renewenvironment{solexo}{\begin{proof}\textcolor{magenta}{\fbox{Please, comment "solexo" in the first lines of defs.tex once the proof is written}}\\}{\end{proof}}


\newenvironment{oldversion}{\ \newline\noindent $\blacktriangleright$ \hrulefill \textcolor{red}{old version: start} \hrulefill $\blacktriangleleft$\newline}{\ \newline\noindent $\blacktriangleright$ \hrulefill \textcolor{red}{old version: end}\hrulefill $\blacktriangleleft$ \newline}

\newenvironment{newversion}{\ \newline\noindent $\blacktriangleright$ \hrulefill \textcolor{blue}{new version: start} \hrulefill $\blacktriangleleft$\newline}{\ \newline\noindent $\blacktriangleright$ \hrulefill \textcolor{blue}{new version: end}\hrulefill $\blacktriangleleft$ \newline}

\newenvironment{encart}[1][]{\ \newline\noindent $\blacktriangleright$ \hrulefill \textcolor{violet}{\textbf{\textsc{#1: start}}} \hrulefill $\blacktriangleleft$\newline}{\ \newline\noindent $\blacktriangleright$ \hrulefill \textcolor{violet}{\textbf{\textsc{ end}}}\hrulefill $\blacktriangleleft$ \newline}

\newenvironment{memo}{\begin{shaded}}{\end{shaded}}



\newenvironment{enumerateList}{ \begin{enumerate}[(i),wide=0pt, labelindent=\parindent]}{\end{enumerate}}


\newenvironment{itemizeList}{ \begin{itemize}[wide=0pt, labelindent=\parindent]}{\end{itemize}}


\def\wdeltafc{W_C}

\def\ddoeblin{contracting}


\newcommand{\tcr}[1]{\textcolor{red}{#1}}

\newcommand{\philippe}[1]{\todo[color=magenta!10]{ Ph: #1}}
\newcommand{\eric}[1]{\todo[color=green!20]{ Er: #1}}
\newcommand{\rd}[1]{\todo[color=blue!20]{ Rd: #1}}
\newcommand{\pierre}[1]{\todo[color=yellow!20]{Pierre: #1}}

\def\card{\mathrm{card}}

%%%%%%%%%%% les notations pour le renouvellement discret

\def\waitingdistr{b}
\def\Waitingdistr{\bar{\waitingdistr}}
\def\waitinggen{B}
\def\delaydistr{a}
\def\delaystation{a_{\mathrm{s}}}
\newcommandx\accesset[1][1=P]{\Xsigma^+_{#1}}
\newcommandx\accessetdiscret[1][1=P]{\Xset^+_{#1}}
\newcommandx\forwardinitdistr[1][1=\delaydistr]{\mu_{#1}}
\def\delaygen{A}
\newcommand\renoup{u} %%%%% pure renewal sequence
\newcommandx\renoud[1][1=]{v_{#1}}  %%%%% delayed renewal sequence
\newcommandx\renoudgen[1][1=]{V_{#1}} %%%%% fonction generatrice de la delayed renewal sequence
\newcommand\renoupgen{U} %%%%% fonction generatrice de la zero-delayed renewal sequence

\newcommand{\paramset}{\Theta}
\newcommand{\param}{\theta}

\renewcommand{\Re}{\mathrm{Re}}
\renewcommand{\Im}{\mathrm{Im}}

\def\bigone{\mathbf{1}}
\newcommandx{\indi}[2][1=]{\1^{#1}_{#2}}
\newcommand{\indiacc}[1]{\1_{\{#1\}}}
\newcommand{\indin}[1]{\1\left\{#1\right\}}

%%%% Linear State Space Notations
\def\corr{\PCor}               % define \corr
\def\rank{\hbox{rank}}
%\def\range{\hbox{range}}
\def\vec{\Vvec}
\def\ssf{{\tt}}
\def\bfit{\em }
\def\bb{{\bm\beta}}
\def\ab{{\bm\alpha}}
\def\pb{{\bm\phi}}
\def\lb{{\bm\lambda}}
\def\rb{{\bm\rho}}
\def\bz{{\bm z}}

\def\mub{{\bm\mu_0}}

\def\gcd{\mathrm{g.c.d.}}



% definition for R stat package% %
\def\R{{\sf R}\ }
% % % % % % % % % % %

% hyperref changes go here
\renewcommand{\subsectionautorefname}{Section}  % these have to come after begin{doc}
\renewcommand{\sectionautorefname}{Section}
\renewcommand{\chapterautorefname}{Chapter}

\newcommand{\argmax}{\operatornamewithlimits{arg\,max}}
\newcommand{\argmin}{\operatornamewithlimits{arg\,min}}

%%% definitions of sets
\def\rset{\ensuremath{\mathbb{R}}}
\def\rsetbar{\ensuremath{\bar{\mathbb{R}}}}
\def\rplus{\ensuremath{\mathbb{R}_+}}
\def\qset{\ensuremath{\mathbb{Q}}}
\def\nset{\ensuremath{\mathbb{N}}}
\def\zset{\ensuremath{\mathbb{Z}}}
\def\cset{\ensuremath{\mathbb{C}}}



%\def\1{\ensuremath{\mathds{1}}}
\def\xset{\ensuremath{\mathcal{X}}}
\def\yset{\ensuremath{\mathcal{Y}}}
\def\bP{\ensuremath{\mathbb{P}}}
\def\bE{\ensuremath{\mathbb{E}}}
\def\calF{\ensuremath{\mathcal{F}}}
\def\calH{\ensuremath{\mathcal{H}}}

\def\mbh{\mathbb{H}} %%% notation pour un espace de Hilbert dans le chapitre Wasserstein
\def\mbz{\zset} %%% notation pour une va dans un  espace de Hilbert dans le chapitre Wasserstein
\def\mbx{\mathbb{X}} %%% notation pour une va dans un  espace de Hilbert dans le chapitre Wasserstein



\newcommand{\distiid}{\ensuremath{\stackrel{\mathrm{iid}}{\sim\,}}}    % ~ with iid over it (for display)
\newcommand{\simiid}{\ensuremath{ {\sim}\,{\rm iid\,} }}         % ~ with iid next to it (for text)
\newcommand{\wn}{\ensuremath{\mathrm{WN}}}   % white noise
\newcommand{\om}{\omega}
\newcommand{\omj}{\omega_j}
\newcommandx{\predx}[3][1=\bm X]{#1_{#2|#3}}   % prediction of boldface X as default
\newcommand{\predxmse}[2]{P_{#1|#2}}  % MSPE
\newcommand{\mslim}{\ensuremath{\stackrel{\text{m.s.}}{\longrightarrow}}}   %mean square convergence
\newcommand{\PCor}{\ensuremath{\operatorname{Cor}}}  % correlation
\newcommand{\cum}{{\rm cum}}
%\newcommand{\mb}{\bm}  % I tend to switch these
% \newcommand{\doublesum}{\mathop{\sum\sum}}
% \newcommand{\triplesum}{\mathop{\sum\sum\sum}}  % used in chapter 2
% \newcommand{\dotsum}{\mathop{\sum\dots\sum}}

\newcommand{\Tr}[1]{\mathrm{Tr}(#1)}
\newcommand{\proj}{\ensuremath{{\sf P\!}}}
\newcommand{\mcy}{\ensuremath{\mathcal Y}}


% % % % % % % % end used in ch1 % % % %

\providecommand{\abs}[1]{\lvert#1\rvert}
\newcommandx\supnorm[2][1=]{| #2 |^{#1}_\infty}
\newcommandx\lnorm[3][1=]{\left\lVert #2 \right\rVert^{#1}_{#3}}
\newcommandx\vectornorm[2][1=]{\left| #2 \right|^{#1}}  %%%% norme de vecteurs pas fonctions
\def\Xinit{\xi} %% initial distribution
\def\borel{\mathcal{B}}
\def\bepsilon{\boldsymbol{\epsilon}}
\def\lleb{\ensuremath{\mathrm{Leb}}}
\def\bx{\mathbf{x}}
\def\by{\mathbf{y}}
\newcommand{\vvec}[1]{\mathbf{#1}}
\newcommand{\VVec}[1]{\mathrm{Vec}(#1)}
\def\bA{\mathbf{A}}
\def\bB{\mathbf{B}}
\def\bX{\mathbf{X}}
\def\bY{\mathbf{Y}}
\def\bZ{\mathbf{Z}}
\def\bM{\mathbf{M}}
\def\bV{\mathbb{V}}


\newcommand{\coint}[1]{\left[#1\right)}
\newcommand{\ocint}[1]{\left(#1\right]}
\newcommand{\ooint}[1]{\left(#1\right)}
\newcommand{\ccint}[1]{\left[#1\right]}
\newcommand{\Id}{\mathrm{I}}


\newcommand{\closure}[1]{\overline{#1}}
\newcommand{\interior}[1]{#1^{o}}
%\newcommand{\cas}{{\:\stackrel{\prob-\mathrm{a.s.}}{\longrightarrow}\:}}
%\newcommand{\cp}{{\:\stackrel{\prob}{\longrightarrow}\:}}
\newcommand{\sigmalg}[1]{\mathcal{#1}}

% two optional arguments, with the package xargs.
\newcommandx\cov[3][1=]{\ensuremath{\mathrm{Cov}_{#1}\left( #2,#3 \right)}}
\newcommandx\var[2][1=]{\ensuremath{\mathrm{Var}_{#1}\left( #2 \right)}}
\newcommandx\cvar[3][1=]{\ensuremath{\mathrm{Var}_{#1}\left( \left. #2 \right| #3 \right)}}
\newcommandx\ccov[3][1=]{\ensuremath{\mathrm{Cov}_{#1}\left( \left. #2 \right| #3 \right)}}


\newcommand{\ci}[4][]%
{%
\ifthenelse{\equal{#1}{}}{\ensuremath{#2 \perp\!\!\!\perp #3 \mid #4 }}{\ensuremath{#2 \perp\!\!\!\perp #3 \mid #4 \; \: [#1]}}%
}

% \haslaw{X}{\mu} math symbol for "$X$ has law $\mu$"
\newcommand{\haslaw}[2]{\mathcal{L}(#1)=#2}
\newcommand{\hascondlaw}[3]{\mathcal{L}(#1|#2)=#3}

%\newcommand{\lip}[1]{\mathrm{L}(#1)}
\def\bu{\mathbf{u}}
%\newcommandx\bprob[2][1=,2=]{\ensuremath{\bar{{\mathbb P}}_{#1}^{#2}}}
\newcommandx\tprob[2][1=,2=]{\ensuremath{\tilde{{\mathbb P}}_{#1}^{#2}}}

\newcommandx\besp[2][1=,2=]{\ensuremath{\bar{{\mathbb E}}_{#1}^{#2}}}
\newcommandx\esp[2][1=,2=]{\ensuremath{{\mathbb E}_{#1}^{#2}}}
\newcommandx\tesp[2][1=,2=]{\ensuremath{\tilde{{\mathbb E}}_{#1}^{#2}}}
\newcommandx\espp[3][1=,2=]{\ensuremath{{\mathbb E}_{#1}^{#2} \left[ #3 \right]}}
\newcommandx\cesp[4][1=,2=]{\ensuremath{{\mathbb E}_{#1}^{#2}\left[ \left. #3 \right| #4 \right]}}


\newcommandx{\cCPE}[3][1=]{{\check{\mathbb{E}}}_{#1}\left[\left. #2 \, \right| #3 \right]} %%%% esperance conditionnelle pour la mesure splitee
\newcommandx{\cCPP}[3][1=]{\check{\mathbb{P}}_{#1}\left(\left. #2 \, \right| #3 \right)}
\def\bell{D} %%%%%% la bell variable pour le splitting
\def\bellv{d}

\newcommandx\cesptilde[4][1=,2=]{\ensuremath{\tilde{{\mathbb E}}_{#1}^{#2}\left[ \left. #3 \right| #4 \right]}}
\newcommandx\cbesp[4][1=,2=]{\ensuremath{\bar{{\mathbb E}}_{#1}^{#2}\left[ \left. #3 \right| #4 \right]}}
\newcommandx\cprob[4][1=,2=]{\ensuremath{{\mathbb P}_{#1}^{#2}\left[ \left. #3 \right| #4 \right]}}
\newcommandx\cprobtilde[4][1=,2=]{\ensuremath{\tilde{{\mathbb P}}_{#1}^{#2}\left( \left. #3 \right| #4 \right)}}

\newcommandx\cprobcheck[4][1=,2=]{\ensuremath{\check{{\mathbb P}}_{#1}^{#2}\left( \left. #3 \right| #4 \right)}}
\newcommandx\cbprob[4][1=,2=]{\ensuremath{\bar{{\mathbb P}}_{#1}^{#2}\left[ \left. #3 \right| #4 \right]}}


\newcommandx\sequence[3][2=,3=]
{\ifthenelse{\equal{#3}{}}{\ensuremath{\{ #1_{#2}\}}}{\ensuremath{\{ #1_{#2}, \eqsp #2 \in #3 \}}}}
\newcommandx\sequencePar[3][2=,3=]
{\ifthenelse{\equal{#3}{}}{\ensuremath{\{ #1({#2})\}}}{\ensuremath{\{ #1({#2}), \eqsp #2 \in #3 \}}}}
%\newcommand{\sequencen}[2]{\ensuremath{\{ #1, \eqsp #2 \}}}
\newcommandx{\sequencen}[2][2=n\in\nset]{\ensuremath{\{ #1, \eqsp #2 \}}}
\newcommandx\dsequence[4][3=,4=]{\ensuremath{\{ (#1_{#3}, #2_{#3}), \eqsp #3 \in #4 \}}}
\newcommand{\setmat}[2]{\mathbb{M}_{#1}(#2)}
\newcommandx{\opernorm}[2][2=]{\interleave#1 \interleave_{{#2}}}
\newcommandx{\opernormp}[2][2=]{\interleave#1 \interleave_{{#2}}}
\newcommand{\BoundedLinear}[1]{\mathcal{BL}(#1)}
\newcommandx\proba[3][1=,3=]{{\mathbb P}_{#1}^{#3} (#2)}

\newcommand{\rmi}{\mathrm{i}}
\newcommand{\rme}{\mathrm{e}}
\newcommand{\rmd}{\mathrm{d}}
\def\diam{\mathrm{diam}}



\newcommand{\lone}{\ensuremath{\mathrm{L}^1}}
\newcommand{\ltwo}{\ensuremath{\mathrm{L}^2}}
\newcommandx{\lp}[1][1=p]{\ensuremath{\operatorname{L}^{#1}}}
\newcommand{\linfty}{\ensuremath{\operatorname{L}^\infty}}

\newcommand{\1}{\ensuremath{\mathbbm{1}}}


\newcommand{\etal}{et al.}
\newcommand{\wrt}{with respect to}
\def\iff{if and only if}
%\newcommand{\wrt}{w.r.t.}
\newcommand{\rhs}{right-hand side}
\newcommand{\lhs}{left-hand side}
\newcommand{\iid}{i.i.d.}
\renewcommand{\-}{\mbox{-}}
\newcommandx{\as}[1][1=\PP]{\ensuremath{#1\, -\mathrm{a.s.}}}
\newcommand{\mae}{\ensuremath{\mathrm{a.e.}}}
\newcommand{\aae}[1]{\ensuremath{#1-\mathrm{a.e.}}}
\newcommand{\aall}[1]{\ensuremath{#1-\mathrm{almost\ all}}}
\newcommand{\ie}{i.e.}
\newcommand{\wlogg}{without loss of generality}
\newcommand{\io}{\ensuremath{\mathrm{i.o.}}}
\newcommand{\ifof}{if and only if}
\newcommand{\eg}{e.g.}
\newcommand{\cf}{cf.}
\newcommand{\pdf}{probability density function}
\newcommand{\eqsp}{\;}
\newcommand{\eqspp}{\ \ }
\newcommand{\eqsppeq}{\quad \quad }


\newcommand{\eqdef}{\ensuremath{:=}}
\newcommand{\eqdefrev}{\ensuremath{=:}}
% \newcommand{\eqdef}{\ensuremath{\stackrel{\mathrm{def}}{=}}}
\def\mca{\mathcal{A}}
\def\mcbb{\mathcal{B}}  %%% \mcb est déjà pris
\def\mcc{\mathcal{C}}
\def\mce{\mathcal{E}}
\def\mcf{\mathcal{F}}
\def\mcg{\mathcal{G}}
\def\mch{\mathcal{H}}
\def\mcm{\mathcal{M}}
\def\mcp{\mathcal{P}}
\def\mcu{\mathcal{U}}
\def\mcv{\mathcal{V}}
\def\shift{\theta}
\def\bshift{\bar{\theta}}
\def\InvariantField{\mathcal{I}}

%%%%%%%%%%%%%
% Probability
\newcommand{\PP}[1][]{\ifthenelse{\equal{#1}{}}{\ensuremath{\mathbb{P}}}{\ensuremath{\mathbb{P}\left( #1 \right)}}}
\newcommand{\bPP}[1][]{\ifthenelse{\equal{#1}{}}{\ensuremath{\bar{\mathbb{P}}}}{\ensuremath{\bar{\mathbb{P}}\left( #1 \right)}}}
\newcommand{\QQ}[1][]{\ifthenelse{\equal{#1}{}}{\ensuremath{\mathbb{Q}}}{\ensuremath{\mathbb{Q}\left( #1 \right)}}}


\newcommandx{\PPRen}[1][1=\delaydistr]{\mathbb{P}_{#1}} %%%% loi d'un renouvellement avec delaydistr et watingdistr
\newcommandx{\PERen}[1][1=\delaydistr]{\mathbb{E}_{#1}} %%%% loi d'un renouvellement avec delaydistr et watingdistr


\newcommand{\PE}[1][]{\ifthenelse{\equal{#1}{}}{\ensuremath{\mathbb E}}{\ensuremath{{\mathbb E}\left[ #1 \right]}}}
\newcommand{\PEs}[1]{\PE{[\, #1\,]}}

\newcommand{\bPE}[1][]{\ifthenelse{\equal{#1}{}}{\ensuremath{\bar{\mathbb E}}}{\ensuremath{\bar{\mathbb E}\left[ #1 \right]}}}
% David changed the command below to the one above
%\newcommand{\PE}[1][]{\ifthenelse{\equal{#1}{}}{\ensuremath{\operatorname{E}}}{\ensuremath{\operatorname{E}\left[ #1 \right]}}}
\newcommandx{\PPdoup}[2][1=,2=]{\mathbb{P}_{#1}^{#2}} % doup comme down-up
\newcommandx{\PEdoup}[2][1=,2=]{\mathbb{E}_{#1}^{#2}}

\newcommandx{\bCPE}[3][1=]
{\bar{{\mathbb E}}_{#1}\left[\left. #2 \, \right| #3 \right]}
\newcommandx{\CPbP}[3][1=]{\bar{{\mathbb P}}_{#1}\left(\left. #2 \, \right| #3 \right)} %%%% proba conditionnelle

\newcommandx{\CPhE}[3][1=]
{\hat{{\mathbb E}}_{#1}\left[\left. #2 \, \right| #3 \right]}
\newcommandx{\CPhP}[3][1=]{\hat{{\mathbb P}}_{#1}\left(\left. #2 \, \right| #3 \right)} %%%% proba conditionnelle
\newcommandx{\CPE}[3][1=]{{\mathbb E}_{#1}\left[\left. #2 \, \right| #3 \right]} %%%% esperance conditionnelle
\newcommand{\CPP}[3][]
{\ifthenelse{\equal{#1}{}}{{\mathbb P}\left(\left. #2 \, \right| #3 \right)}{{\mathbb P}_{#1}\left(\left. #2 \, \right | #3 \right)}}
\newcommand{\bCPP}[3][]
{\ifthenelse{\equal{#1}{}}{{\bar{\mathbb P}}\left(\left. #2 \, \right| #3 \right)}{{\bar{\mathbb P}}_{#1}\left(\left. #2 \, \right | #3 \right)}}

\newcommand{\barP}{\bar{P}}
\newcommand{\barV}{\bar{V}}
\newcommand{\barC}{\bar{C}}
\newcommand{\barcc}{\bar{C}^c}
\newcommand{\barb}{\bar{b}}




\newcommandx{\PVar}[1][1=]{\ensuremath{\operatorname{Var}_{#1}}}
\newcommand{\CPVar}[2]{\ensuremath{\operatorname{Var}\left(\left. #1 \, \right| #2 \right)}}
\newcommandx{\PCov}[1][1=]{\ensuremath{\operatorname{Cov}_{#1}}}
\newcommand{\PCorr}{\ensuremath{\operatorname{Corr}}}

%%%% les lois
\newcommand{\gauss}{\ensuremath{\operatorname{N}}}
\newcommand{\gammadist}{\ensuremath{\operatorname{\Gamma}}}
\newcommand{\cauchy}{\ensuremath{\operatorname{C}}}
\newcommand{\lognorm}{\ensuremath{\operatorname{LN}}}
\newcommand{\dirich}{\ensuremath{\operatorname{Dir}}}
\newcommand{\negexp}{\ensuremath{\operatorname{Exp}}}
\newcommand{\gam}{\ensuremath{\operatorname{Ga}}}
\newcommand{\invgam}{\ensuremath{\operatorname{IG}}}
\newcommand{\unif}{\ensuremath{\operatorname{Unif}}}
\newcommand{\bin}{\ensuremath{\operatorname{Bin}}}
\newcommand{\betad}{\ensuremath{\operatorname{Be}}}
\newcommand{\poiss}{\ensuremath{\operatorname{Pn}}}
\newcommand{\mult}{\ensuremath{\operatorname{Mult}}}
\newcommandx{\ber}[1][1=\epsilon]{\ensuremath{\operatorname{b}_{#1}}}




\newcommand\distgeneric{\boldsymbol\rho}
\newcommand{\tvdistsym}{\ensuremath{\mathrm{d}_{\mathrm{TV}}}}
\newcommandx{\tvdist}[3][1=]{\ensuremath{\mathrm{d}^{#1}_{\mathrm{TV}}}(#2,#3)}
\newcommand{\tvnorm}[1]{\ensuremath{\left\|#1\right\|_{\mathrm{TV}}}}
\newcommand{\Vnorm}[2]{\ensuremath{\left\|#1\right\|_{\mathrm{#2}}}}
\newcommand{\VnormMeas}[2]{\ensuremath{\left\|#1\right\|_{#2}}}
\newcommandx{\VnormFunc}[3][1=]{\ensuremath{\left|#2\right|_{{#3}}^{#1}}}
\newcommand{\blnorm}[1]{\ensuremath{\left\|#1\right\|_{\mathrm{BL}}}}
\newcommand{\dbl}[2]{\ensuremath{d_{\mathrm{BL}}(#1,#2)}}
\newcommandx{\oscnorm}[3][1=,3=]%
{\operatorname{osc}^{#1}_{#3}\left(#2\right)}


\def\neighborhood{\mathcal{V}}

%%%%%%%%%%%%%%%%%%%%% class of functions
% ARGUMENT DE TYPE X
\newcommandx\functionset[3][1=,3=]{
\ifthenelse{\equal{#1}{c}}
{\mathrm{C}(\mathsf{#2})}%fonctions continues
{\ifthenelse{\equal{#1}{bc}}{\mathrm{C}^{#3}_b(\mathsf{#2})}%fonctions continues born\'{e}es
{\ifthenelse{\equal{#1}{u}}{\mathrm{U}^{#3}(\mathsf{#2})}%fonctions uniform\'{e}ment continues
{\ifthenelse{\equal{#1}{bu}}{\mathrm{U}_b^{#3}(\mathsf{#2})}%fonctions uniform\'{e}ment continues born\'{e}es
{\ifthenelse{\equal{#1}{l}}{\mathrm{Lip}_{#3}(\mathsf{#2})}%fonctions lipschitz
{\ifthenelse{\equal{#1}{cz}}{\mathrm{C}_0^{#3}(\mathsf{#2})}
{\ifthenelse{\equal{#1}{k}}{\mathrm{C}_c^{#3}(\mathsf{#2})} %fonctions continues à support compact
{\ifthenelse{\equal{#1}{bl}}{\mathrm{Lip}_{b#3}(\mathsf{#2})}%fonctions lipschitz born\'{e}es
{\ifthenelse{\equal{#1}{ldp}}{\mathrm{Lip}_{#3}(\mathsf{#2})}%fonctions lipschitz à poids
%{\mathbb{F}_{#1}^{#3}(\mathsf{#2},\mathcal{#2})}%le reste
{\mathbb{F}_{#1}^{#3}(\mathsf{#2})}%le reste, j imagine que la ligne du dessus comporte un \mathcal{#2} de trop...
}}}}}}}}}

\newcommand{\functionsetalone}[3]{\mathbb{F}_{#1}\lr{#2,#3}}

% IDEM, ARGUMENT DE TYPE \Xset,\borel(\Xset)
\newcommandx\functionsetarg[2][1=]{\ifthenelse{\equal{#1}{c}}{\mathrm{C}(\mathsf{#2})}%fonctions continues
{\ifthenelse{\equal{#1}{bc}}{\mathrm{C}_b(#2)}%fonctions continues born\'{e}es
{\ifthenelse{\equal{#1}{u}}{\mathrm{U}(#2)}%fonctions uniform\'{e}ment continues
{\ifthenelse{\equal{#1}{bu}}{\mathrm{U}_b(#2)}%fonctions uniform\'{e}ment continues born\'{e}es
{\ifthenelse{\equal{#1}{cz}}{\mathrm{C}_0(#2)}
{\ifthenelse{\equal{#1}{k}}{\mathrm{C}_c(#2)}%fonctions continues à support compact
{\ifthenelse{\equal{#1}{l}}{\mathrm{Lip}(#2)}%fonctions lipschitz
{\ifthenelse{\equal{#1}{bl}}{\mathrm{Lip}_b(#2)}%fonctions lipschitz born\'{e}es
{\mathbb{F}_{#1}(#2)}%le reste
}}}}}}}}

%IDEM, SANS ARGUMENT
\newcommandx\functionsetspec[1][1=]{
\ifthenelse{\equal{#1}{c}}{\mathrm{C}}%fonctions continues
{\ifthenelse{\equal{#1}{bc}}{\mathrm{C}_b}%fonctions continues born\'{e}es
{\ifthenelse{\equal{#1}{u}}{\mathrm{U}}%fonctions uniform\'{e}ment continues
{\ifthenelse{\equal{#1}{bu}}{\mathrm{U}_b}%fonctions uniform\'{e}ment continues born\'{e}es
{\ifthenelse{\equal{#1}{l}}{\mathrm{Lip}}%fonctions lipschitz
{\ifthenelse{\equal{#1}{bl}}{\mathrm{Lip}_b}%fonctions lipschitz born\'{e}es
{\mathbb{F}_{#1}}%le reste
}}}}}}

%%%%%%%%%%  espaces Lp
\newcommand\lpspace[2]{\mathcal{L}^{#1}(#2)} %%%% espace $\mathcal{L}^p(\mu)$

% abreviations of lipschitz, bounded lipschitz
\newcommand{\lipabb}{\mathrm{Lip}}
\newcommandx{\blabb}[1][1=\distance]{\mathrm{BL}(#1)}
\newcommandx{\normLip}[2][1=\distance]{\left|#2\right|_{\mathrm{Lip}(#1)}}
\newcommandx{\normBL}[2][1=\distance]{\left|#2\right|_{\mathrm{BL}(#1)}}
% Set of Lipschitz functions on X: \functionset[\lipabb]{X}, bounded lipschitz:\functionset[\blabb]{X}
%\newcommandx\lipshitzset[3][1=]{\mathbb{L}_{#1}(#2,#3)}

\def\radon{\mathrm{r}}
\newcommandx\measureset[3][1=\mathrm{s},3=]{\mathbb{M}^{#3}_{#1}(\mathcal{#2})}
\newcommandx\measuresetmetric[2][1=1]{\mathbb{M}_{#1}(\mathcal{B}(\mathsf{#2}))}  %%% espace de mesures sur un metric muni de sa tribu borelienne
\newcommandx\measuresetspec[1][1=\mathrm{s}]{\mathbb{M}_{#1}}
\newcommandx\measuresetarg[3][1=\mathrm{s},3=]{\mathbb{M}_{#3}^{#1}(#2)}
\newcommandx\ball[3][1=]{\mathrm{B}_{#1} (#2,#3)}


\newcommandx\bernoulli[1][1=\epsilon]{\operatorname{Be}({#1})}


\newcommandx\spacexd[3][1=\distance,3=]{\mathbb{S}_{#3}(\mathsf{#2},#1)}

%%%%%%%%%%%%%% used in Chap erg Ele%%%%%%%%%%%%%%%%%%%%%
%%%%%%%%%%%%%% NORMS ON MEASURES   %%%%%%%%%%%%%%%%%%%%%
\newcommandx{\wassnorm}[2][1]{\lVert #2\rVert_{#1}}

\newcommand{\marginal}[2]{\mathcal{M}\left(#1,#2\right)}
\newcommand\atom{\alpha}

\newcommandx{\Vdist}[3][1=V]{\mathrm{d}_{#1}(#2,#3)}
\newcommandx{\Vdistvide}[1][1=V]{\mathrm{d}_{#1}}
\newcommandx{\wassersym}[1][1=\distance]{\mathbf{W}_{#1}}
\newcommandx{\wasser}[4][1=\distance,4=]{\mathbf{W}_{#1}^{#4}\left(#2,#3\right)}
\newcommand{\hammingdist}{\distance_H}
\newcommandx{\prohosym}[1][1=\distance]{{\bm\rho}_{#1}}
\newcommandx{\proho}[4][1=\distance,4=]{{\bm\rho}_{#1}^{#4}\left(#2,#3\right)}
\newcommandx{\dualblsym}[1][1=\distance]{{\bm\beta}_{#1}}
\newcommandx{\dualbl}[4][1=\distance,2=]{{\bm\beta}^{#2}_{#1}\left(#3,#4\right)}
\newcommandx{\duallipsym}[1][1=\distance]{{\bm\gamma}_{#1}}
\newcommandx{\duallip}[4][1=\distance,2=]{{\bm\gamma}^{#2}_{#1}\left(#3,#4\right)}
\def\couplingmeasure{\mathcal{C}}

\newcommandx{\pscal}[3][3=]{\ifthenelse{\equal{#3}{}}{\left\langle #1, #2 \right\rangle}{\left\langle #1, #2 \right\rangle_{\operatorname{L}^2(#3)}}}
\newcommandx{\mscal}[3][3=]{\ifthenelse{\equal{#3}{}}{\left( #1, #2 \right)}{\left( #1, #2 \right)_{{\measuresetarg[2]{#3}}}}}

\newcommandx{\dualnormBL}[2][1=d]{\|#2\|_{\mathrm{BL}(#1)}}

\newcommand{\dobrush}{\mathsf{\Delta}}
\newcommandx{\dobru}[3][1=,3=]{\dobrush_{#1}^{#3}\left( #2\right)}  %%% dobrushin coefficient
\newcommandx{\tildedobru}[3][1=,3=]{\tilde{\dobrush}_{#1}^{#3}\left( #2\right)}  %%% modified dobrushin coefficient, for the proof

\newcommand{\distmeasureset}[2]{\mathbb{S}(#1,#2)}
\newcommand{\Uset}{\mathrm U}
\newcommand{\Usigma}{\mathcal U}
\newcommand{\dist}{\Upsilon}

%%%%%%%%%%%%% end notations used in chap erg ele %%%%%%%%%


%%%% probability convergence
\newcommand{\weaklim}{\ensuremath{\stackrel{{w}}{\Rightarrow}}}
\newcommand{\starweaklim}{\ensuremath{\stackrel{\textrm{w}^*}{\Rightarrow}}}
\newcommand{\vaguelim}{\ensuremath{\stackrel{\textrm{vague}}{\Rightarrow}}}

\newcommand{\eqd}{\ensuremath{\stackrel{\mathrm{law}}{=}}}   %%%% égalité en loi
\newcommandx{\plim}[1]{\ensuremath{\stackrel{#1\mbox{-}\text{prob}}{\longrightarrow}}}
\newcommandx{\dlim}[1]{\ensuremath{\stackrel{#1}{\Longrightarrow}}}
\newcommandx{\aslim}[1]{\ensuremath{\stackrel{#1 \mbox{-} \text{a.s.}}{\longrightarrow}}}

\newcommand{\mcb}[1]{\ensuremath{\mathcal{B}\left(#1\right)}}
\newcommand{\existLim}[1][]{\underset{#1}{\rightsquigarrow}} % means that the limit exists

%%%% Ergodicity and mixing
\def\opinv{\ensuremath{\operatorname{T}}}
\def\invarmes{\ensuremath{\mathbb{P}}}
\def\systdyn{\ensuremath{(\Omega, \mathcal{B}, \invarmes, \opinv)}}
\def\probaspace{\ensuremath{(\Omega, \mathcal{B}, \invarmes)}}


%%%%%%%%%%%%%%%
% Markov Chain and HMM notations
% Hidden states and observations
\newcommand{\X}{\ensuremath{X}}
\newcommand{\Y}{\ensuremath{Y}}
\newcommand{\x}{\ensuremath{x}}
\newcommand{\Aset}{\ensuremath{\mathsf{A}}}
\newcommand{\Cset}{\ensuremath{\mathbb{C}}}
\newcommand{\Xset}{\ensuremath{\mathsf{X}}}

\newcommand{\Asigma}{\ensuremath{\mathcal{A}}}
\newcommand{\Xsigma}{\ensuremath{\mathcal{X}}}
\newcommand{\Eset}{\ensuremath{\mathsf{E}}}
\newcommand{\Esigma}{\ensuremath{\mathcal{E}}}
\newcommand{\Hset}{\ensuremath{\mathsf{H}}}
\newcommand{\Hsigma}{\ensuremath{\mathcal{H}}}

\newcommand{\Xchain}{\ensuremath{\sequence{X}[k][\nset]}}
\newcommand{\Xproc}{\ensuremath{\sequence{X}[k][\nset]}}
\newcommand{\Xprocshort}{\ensuremath{\{X_k\}}}
\newcommand{\Yset}{\ensuremath{\mathsf{Y}}}
\newcommand{\Ysigma}{\ensuremath{\mathcal{Y}}}
\newcommand{\Yproc}{\ensuremath{\{Y_k\}_{k\geq 0}}}
\newcommand{\Yprocshort}{\ensuremath{\{Y_k\}}}
\newcommand{\Yprocone}{\ensuremath{\{Y_k\}_{k\geq 1}}} % for order chapter
\newcommand{\XYchain}{\ensuremath{\{X_k,Y_k\}_{k\geq 0}}}
\newcommand{\XYchainshort}{\ensuremath{\{X_k,Y_k\}}}
\newcommand{\XYproc}{\ensuremath{\{X_k,Y_k\}_{k\geq 0}}}
\newcommand{\XYprocshort}{\ensuremath{\{X_k,Y_k\}}}
\newcommand{\Zset}{\ensuremath{\mathsf{Z}}}
\newcommand{\Zsigma}{\ensuremath{\mathcal{Z}}}
\newcommand{\Tset}{\ensuremath{\mathsf{T}}}
\newcommand{\Tsigma}{\ensuremath{\mathcal{T}}}



\newcommand{\crochet}[1]{\ensuremath{\langle #1 \rangle}}


\newcommand{\chunk}[4][]%
{\ifthenelse{\equal{#1}{}}{\ensuremath{{#2}_{#3:#4}}}{\ensuremath{#2^#1}_{#3:#4}}
}
%%%%%%%%%%%%%%
% Filtering and Smoothing notations
\newcommand{\pred}[3][]%
{%
\ifthenelse{\equal{#1}{}}{\ensuremath{\phi_{#2|#3}}}{\ensuremath{\phi_{#1,#2|#3}}}%
}
\newcommand{\adjfunc}[4][]
{\ifthenelse{\equal{#1}{}}{\ifthenelse{\equal{#4}{}}{\vartheta_{#2}}{\vartheta_{#2}(#4)}}
{\ifthenelse{\equal{#1}{smooth}}{\ifthenelse{\equal{#4}{}}{\tilde{\vartheta}_{#2}}{\tilde{\vartheta}_{#2}(#4)}}
{\ifthenelse{\equal{#1}{fully}}{\ifthenelse{\equal{#4}{}}{\vartheta^*_{#2}}{\vartheta^*_{#2}(#4)}}{\mathrm{erreur}}}}}



\newcommand{\smwght}[3]{\tilde{\omega}_{#1|#2}^{#3}}
\newcommand{\smwghtfunc}[2]{\tilde{\omega}_{#1|#2}}


\newcommand{\post}[3][]%
{
\ifthenelse{\equal{#1}{}}{\ensuremath{\phi_{#2|#3}}}%
{\ifthenelse{\equal{#1}{hat}}{\ensuremath{\phi^{N}_{#2|#3}}}
{\ifthenelse{\equal{#1}{tilde}}{\ensuremath{\tilde{\phi}^{N}_{#2|#3}}}
{\ifthenelse{\equal{#1}{tar}}{\ensuremath{\phi^{N,\mathrm{t}}_{#2|#3}}}}
}
}
}

\newcommand{\filt }[2][]%
{
\ifthenelse{\equal{#1}{}}{\ensuremath{\phi_{#2}}}%
{\ifthenelse{\equal{#1}{hat}}{\ensuremath{\phi^{N}_{#2}}}
{\ifthenelse{\equal{#1}{tilde}}{\ensuremath{\tilde{\phi}^{N}_{#2}}}
{\ifthenelse{\equal{#1}{tar}}{\ensuremath{\phi^{N,\mathrm{t}}_{#2}}}
{\ifthenelse{\equal{#1}{aux}}{\ensuremath{\phi^{N,\mathrm{a}}_{#2}}}
}
}
}
}
}

\newcommand{\unfilt}[2][]%
{
\ifthenelse{\equal{#1}{}}{\ensuremath{\gamma_{#2}}}%
{\ifthenelse{\equal{#1}{hat}}{\ensuremath{\gamma^{N}_{#2}}}
{\ifthenelse{\equal{#1}{tilde}}{\ensuremath{\tilde{\gamma}^{N}_{#2}}}
{\ifthenelse{\equal{#1}{tar}}{\ensuremath{\gamma^{N,\mathrm{t}}_{#2}}}
{\ifthenelse{\equal{#1}{aux}}{\ensuremath{\gamma^{N,\mathrm{a}}_{#2}}}
}
}
}
}
}

\newcommand{\bPhi}{\boldsymbol{\Phi}}
\newcommand{\bespilon}{\boldsymbol{\epsilon}}

\def\cXset{\check{\Xset}}
\def\cXsigma{\check{\Xsigma}}
\def\calpha{{\check{\alpha}}}
\def\cP{\check{P}}
\def\cPP{\check{\PP}}
\def\cPE{\check{\PE}}

\newcommand{\lagvector}[3]{#1_{#2},\dots,#1_{#3}}



%%%%%%%%%%%%%% Particle Filtering Notations
\newcommand{\epart}[2][]
{%
\ifthenelse{\equal{#1}{}}{\ensuremath{\xi^{#2}}}{\ensuremath{\xi^{#2}_{#1}}}
}
\newcommand{\etpart}[2][]
{%
\ifthenelse{\equal{#1}{}}{\ensuremath{\tilde{\xi}^{#2}}}{\ensuremath{\tilde{\xi}^{#2}_{#1}}}
}
\newcommand{\etwght}[2][]{%
\ifthenelse{\equal{#1}{}}{\ensuremath{\tilde{\omega}^{#2}}}{\ensuremath{\tilde{\omega}^{#2}_{#1}}}
}
\newcommand{\ewght}[2][]{%
\ifthenelse{\equal{#1}{}}{\ensuremath{\omega^{#2}}}{\ensuremath{\omega^{#2}_{#1}}}
}
\newcommand{\NISE}[4][]%
{%
\ifthenelse{\equal{#1}{}}{\ensuremath{\tilde{#2}^{\scriptstyle \mathrm{IS}}_{#4}\left(#3 \right)}}{\ensuremath{\tilde{#2}^{\scriptstyle \mathrm{IS}}_{#1,#4}\left(#3 \right)}}%
}
\newcommand{\ISE}[4][]%
{%
\ifthenelse{\equal{#1}{}}{\ensuremath{\widehat{#2}^{\scriptstyle  \mathrm{IS}}_{#4} \left(#3 \right)}}{\ensuremath{\widehat{#2}^{\scriptstyle  \mathrm{IS}}_{#1,#4} \left(#3 \right)}}%
}
\newcommand{\SIRE}[4][]%
{%
\ifthenelse{\equal{#1}{}}{\ensuremath{\hat{#2}^{\scriptstyle  \mathrm{SIR}}_{#4} \left(#3 \right)}}{\ensuremath{\hat{#2}^{\scriptstyle  \mathrm{SIR}}_{#1,#4} \left(#3 \right)}}%
}
\newcommand{\MCE}[3]%
{
{\ensuremath{\hat{#1}^{\scriptstyle  \mathrm{MC}}_{#3} \left(#2 \right)}}%
}
\newcommand{\KUN}{\ensuremath{Q}}
\newcommand{\kun}{\ensuremath{\ell}}
\newcommand{\KISS}{\ensuremath{R}}
\newcommand{\kiss}{\ensuremath{r}}
\newcommand{\XinitTAR}{\ensuremath{\phi_{0}}}
\newcommand{\TAR}[1]{\ensuremath{\phi_{0:#1|#1}}}
\newcommand{\lhood}[2][]%
{%
\ifthenelse{\equal{#1}{}}{\ensuremath{\mathrm{L}_{#2}}}{\ensuremath{\mathrm{L}_{#1,#2}}}%
}



\newcommand{\XinitIS}{\ensuremath{\rho_0}}
\newcommand{\ISD}[1]{\ensuremath{\rho_{0:#1}}}
\newcommand{\ETAR}[1]{\ensuremath{\hat{\phi}_{0:#1|#1}}}
\newcommandx\weightfunc[1][1=]{\operatorname{w}_{#1}}

% Optimal transition kernel
\newcommand{\KOPT}{\ensuremath{T}}
\newcommand{\kopt}{\ensuremath{t}}
% Normalizing constant of the optimal kernel
\newcommand{\NormOPT}{\ensuremath{\gamma}}
\newcommand{\URoot}{\ensuremath{R}}
\newcommand{\VRoot}{\ensuremath{S}}
\newcommand{\VCov}[1][]%
{%
\ifthenelse{\equal{#1}{}}{\VRoot \VRoot^t}{\VRoot_{#1} \VRoot^t_{#1}}%
}
\newcommand{\KGF}[1]{\ensuremath{K_{#1}}}
% Coefficients of variation
\def\CV{\mathrm{CV}}
\newcommand{\ebwght}[2]{\ensuremath{\bar{\omega}_{#1}^{#2}}}
\newcommand{\proppart}[2]{\ensuremath{\tilde{\xi}_{#1}^{#2}}}
\newcommand{\PROPTAR}[1]{\ensuremath{\tilde{\phi}_{0:#1|#1}}}
\newcommand{\sumweight}[2][]{%
\ifthenelse{\equal{#1}{}}{\ensuremath{\Omega_{#2}}}{\ensuremath{\Omega_{#2}^{(#1)}}}}
\newcommand{\epartset}{\ensuremath{\mathsf{E}}}
\newcommand{\epartsigma}{\ensuremath{\mathcal{E}}}
\newcommand{\etpartset}{\ensuremath{\mathsf{\tilde{E}}}}
\newcommand{\etpartsigma}{\ensuremath{\mathcal{\tilde{E}}}}
\def\stdnormfunc{\sigma}
\def\LTAR{Q}
\newcommand\ntimes[1]{\eta_{#1}}
\newcommand\ntimesres[1]{H_{#1}}
\newcommand{\partfrac}[1]{\left\langle #1 \right\rangle}
\newcommand{\mcfpart}[2][]{%
\ifthenelse{\equal{#1}{}}{{\mathcal{F}_{#2}}}{\mathcal{F}_{#1}^{#2}}%
}
\newcommand{\Afunc}{\ensuremath{a}}
\newcommand{\Bfunc}{\ensuremath{b}}
\newcommand{\PX}{\ensuremath{W}}
%\newcommand{\px}{\ensuremath{w}}
\newcommand{\PXset}{\ensuremath{\mathsf{W}}}
\newcommand{\PXsigma}{\ensuremath{\mathcal{W}}}
\newcommand{\DX}{\ensuremath{I}}
\newcommand{\dx}{\ensuremath{i}}
\newcommand{\DXset}{\ensuremath{\mathsf{I}}}
\newcommand{\DXsigma}{\ensuremath{\mathcal{I}}}
\newcommand{\order}{\ensuremath{r}}
\newcommand{\tsumweight}[2][]{%
\ifthenelse{\equal{#1}{}}{\ensuremath{\widetilde{\Omega}_{#2}}}{\ensuremath{\widetilde{\Omega}_{#2}^{(#1)}}}}

\newcommand{\BK}[1]{\mathrm{B}_{#1}}

\renewcommand{\aleph}{M_0}
\renewcommand{\beth}{M_1}


\newcommand{\mynegspace}{\hspace{-0.12em}}
\newcommand{\lvvvert}{\rvert\mynegspace\rvert\mynegspace\rvert}
\newcommand{\rvvvert}{\rvert\mynegspace\rvert\mynegspace\rvert}
\newcommand{\triple}[1]{\lvvvert#1\rvvvert}
\renewcommandx{\ker}[1][1=]{\mathrm{Ker}^{#1}}
\newcommand{\vect}{\mathrm{Span}}
\newcommandx{\im}[1][1=]{\mathrm{Im}^{#1}}
\newcommand{\operp}{\oplus}


\newcommandx{\rtis}[2][1=m]{\sigma_{#2,#1}}
\def\rti{\sigma}
\def\hti{\tau}
\def\visite{\tau}

\newcommand{\exit}[1]{T_{#1}}

\def\deltaK{d_{\cal K}}

\newcommand{\ensemble}[2]{\left\{#1\,:\eqsp #2\right\}}
\newcommand{\set}[2]{\ensemble{#1}{#2}}

\newcommand{\projo}{\Phi}

\newcommand{\lsc}{lower semi-continuous}
\newcommand{\usc}{upper semi-continuous}
\newcommand{\supp}{\mathrm{supp}}


\newcommand{\minmeas}[2]{#1 \wedge #2}
\newcommand{\maxmeas}[2]{#1 \vee #2}
\newcommandx{\law}[2][1=\PP]{\mathcal{L}_{#1}\left(#2\right)}

\newcommand\diagonal{\Delta} %%%% diagonale de l'espace produit $\Xset\times\Xset$; notation à  changer!!!

\def\sign{\mathrm{sign}}

\newcommand{\done}{\textcolor{red}{$\blacktriangleright$ Done!}}


\newcommandx\stochdom[1][1=\PP]{ \unlhd_{#1} }


\newcommand\weakstarconv{weak*}
\newcommandx{\weakstar}[1][1=weak]{{#1}*}


%%%%%%%%%%%%%%%%%%%%% sup essentiel

\newcommand\esssup[2]{\left\Vert #2 \right\Vert_{\infty,#1}}
\newcommandx\lpnorm[3][1=p]{\left\Vert #3 \right\Vert_{\operatorname{L}^{#1}(#2)}}
\newcommandx\mpnorm[3][1=p]{\left\Vert #3 \right\Vert_{{\measuresetarg[#1]{#2}}}}

%%%%%%%%% parties positives

\newcommand\pospart[1]{#1^+}
\newcommand\negpart[1]{#1^-}
\newcommand\logplus{\pospart{\log}}
\newcommand\decreasing{decreasing\ }
\newcommand\increasing{increasing\ }
\newcommand\nondecreasing{non decreasing}
\newcommand\nonincreasing{non increasing}

\renewcommand{\mod}[2]{#1\, [#2]}
\newcommand{\alert}[1]{\textcolor{red}{#1}}
\newcommandx{\Sfr}[2][2=k]{\sum_{#2=0}^{#1-1} r(#2) f(X_{#2})}
\newcommandx{\taboo}[3][1=,3=]{\leftidx{_#1^{#3}}{#2}}
\newcommand{\rate}{\beta}
\newcommand{\trate}{\tilde{\beta}}




%%%% sommes partielles
%\newcommand\prim[1]{#1^0} on n a plus besoin de cette commande car j ai introduis la commande \rzero

\newcommand{\lr}[1]{\left( #1 \right)}
\newcommand{\lrb}[1]{\left[ #1 \right]}
\newcommand{\lrcb}[1]{\left\{ #1 \right\}}
\newcommand{\rzero}{r^0}
\newcommand{\rone}{r^*}


\newcommand\inftyset[1]{#1_+}
\newcommand\inftynb[1]{#1_\infty}


\newcommand\rseq{r}
%%%%%%%%%%%%%%%%%%%%%%%%%%%%%%%%%%%%%%%%%
%%%% mesures pour la formule de kac

\def\muczero{\mu_C^0}
\def\mucun{\mu_C^1}
\newcommand{\mukac}[3]{#1_#2^#3}   %%%% ex: $\mu_g^0$

%%%%%%%%%%% suites sous géométriques

\def\quasistationaireseq{\Lambda_1}
\def\autoanihilseq{\Lambda_2}
\def\bautoanihilseq{\bar{\Lambda}_2}
\def\logsubadditiveseq{\mathcal{S}}
\def\blogsubadditiveseq{\bar{\mathcal{S}}}
\def\stonewaingerseq{\Lambda_0}



\newcommandx\multidrift[5][1={\{V_n\}},2=f,3=r,4=b,5=C]{\ensuremath{\operatorname{D}_{\mathrm{SG}}(#1,#2,#3,#4,#5)}} %%% sequence of drift conditions

\newcommandx\subgeometricdrift[4][1=V,2=\phi,3=b,4=C]{\ensuremath{\operatorname{D}_{\mathrm{SG}}(#1,#2,#3,#4)}} %%% subgeometric drift condition
\newcommandx\subgeometricdriftwset[3][1=V,2=\phi,3=b]{\ensuremath{\operatorname{D}_{\mathrm{SG}}(#1,#2,#3)}} %%% subgeometric drift condition



\newcommandx\geometricdrift[4][1=V,2=\lambda,3=b,4=C]{\ensuremath{\operatorname{D}_{\mathrm{G}}(#1,#2,#3,#4)}} %%% subgeometric drift condition
\newcommandx\geometricdriftwset[3][1=V,2=\lambda,3=b]{\ensuremath{\operatorname{D}_{\mathrm{G}}(#1,#2,#3)}} %%% geometric drift condition
% \newcommandx*\multdrift[5][1={\{V_n\}},2=f,3=r,4=b,5=C,usedefault]{\ensuremath{D(#1,#2,#3,#4,#5)}}
%%% la meme chose pour remplacer les valeurs par defaut par [] lorsqu'on en change seulement quelques unes


\def\wfdeltac{W^{f,\delta}_C} %%%%% solution minimale du drift geometrique
\newcommand{\wfrc}[1]{W_{#1,C}^{\tiny{f,r}}} %%%%% solution minimale du drift sous-geometrique

\def\atomd{\mathsf{a}}
\def\atomdb{\mathsf{b}}

\newcommand{\travaux}{\rd{Travaux en cours}}
\newcommand{\restker}[1]{Q_{#1}}
\newcommand{\restmes}[2]{#1\mid_{#2}}
\newcommand{\mustar}{\mu^*}

\def\barf{\bar{f}_\epsilon}

%%%%%%%%%%%%% maximal coupling
\newcommand{\restric}[2]{\left(#1\right)_{#2}}
\newcommand{\add}[2]{_{#2}^{\langle #1\rangle}} % attribut

\def\spec{\operatorname{spec}}
\def\monplus{\stackrel{\bot}{\oplus}}
\def\spectradius{\mathrm{Sp.Rad.}}
\def\range{\operatorname{range}}
\def\distance{\mathrm{{d}}}
\def\cost{\mathrm{{c}}}

%%%%%%%%%%% newcommandes polyXtremes

\newcommandx\invcdf[1][1=X]{F_{#1}^{\leftarrow}}
\newcommandx\cdf[1][1=X]{F_{#1}}
\newcommandx\compcdf[1][1=X]{\overline{F}_{#1}}
\newcommand{\calS}{\mathcal S}
\newcommand{\calC}{\mathcal C}
\newcommand{\calB}{\mathcal B}
\newcommand{\eqlaw}{\stackrel{\mathcal L}{=}}
\newcommand{\eqLaw}{\stackrel{\mathcal L}{=}}
\newcommand{\Xn}[1]{X_{(#1,n)}}
\newcommand{\Un}[1]{U_{(#1,n)}}
\newcommand{\Vn}[1]{V_{(#1,n)}}
\newcommand{\indep}{\rotatebox[origin=c]{90}{$\models$}}
\newcommand{\mcS}{\mathcal S}
\newcommand{\ma}[1]{\hat{#1}}
\newcommand{\mi}[1]{\check{#1}}
\newcommandx\odr[3][1=X,3=n]{#1_{(#2,#3)}}
\def\mcL{{\mathcal L}_0}
\def\mcl{{\mathcal L}}
\newcommand{\domn}[1]{{\mathcal D}\lr{#1}}
\newcommand{\nom}[1]{\hfill \mbox{ $(\blacktriangleright$\textsc{#1}$)$}}
\newcommand{\sgn}{\mathrm{sgn}}
\newcommand{\Lmiss}{L_{\mathrm{miss}}}

\newcommand{\Mplus}{M^+_{d}}
\renewcommand{\det}{\mathrm{det}}




