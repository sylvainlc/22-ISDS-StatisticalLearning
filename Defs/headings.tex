\documentclass[12pt,graybox,envcountchap,envcountsame,sectrefs]{svmono}
\usepackage[Lenny]{fncychap}
\usepackage[utf8]{inputenc}
\usepackage{mathptmx,helvet,type1cm,xargs,srcltx,latexsym,bm,wrapfig,subfigure,multicol,stmaryrd,fancybox}
\usepackage{cancel,a4wide,bbm,graphicx,ushort,amssymb,amsmath,ntheorem,ifthen,amsfonts,bbm,graphicx,twoopt,verbatimbook,color,framed,marvosym,phaistos}
\usepackage[bottom]{footmisc} % places footnotes at page bottom
\usepackage[shortlabels]{enumitem}
\usepackage[title,titletoc,header]{appendix}          % % %  appendices % % % % %
\usepackage{leftidx} % permet les subscript a gauche avec la commande \leftidx{-A}{P}{^k}
\usepackage{tikz}
%\usepackage{a4wide}
\usepackage{answers}
\usepackage{listings}
\usepackage{fancyvrb} 
\lstset{% setup listings 
        language=R,% set programming language 
        basicstyle=\small,% basic font style 
        keywordstyle=\bfseries,% keyword style 
        commentstyle=\ttfamily\itshape,% comment style 
        numbers=left,% display line numbers on the left side 
        numberstyle=\scriptsize,% use small line numbers 
        numbersep=10pt,% space between line numbers and code 
        tabsize=3,% sizes of tabs 
        showstringspaces=false,% do not replace spaces in strings by a certain character 
        captionpos=b,% positioning of the caption below 
        breaklines=true,% automatic line breaking 
        escapeinside={(*}{*)},% escaping to LaTeX 
        fancyvrb=true,% verbatim code is typset by listings 
        extendedchars=false,% prohibit extended chars (chars of codes 128--255) 
        literate={"}{{\texttt{"}}}1{<-}{{$\leftarrow$}}1{<<-}{{$\twoheadleftarrow$}}1 
        {~}{{$\sim$}}1{<=}{{$\le$}}1{>=}{{$\ge$}}1{!=}{{$\neq$}}1{^}{{$^\wedge$}}1,% item to replace, text, length of chars 
        alsoletter={.<-},% becomes a letter 
        alsoother={$},% becomes other 
        otherkeywords={!=, ~, $, *, \&, \%/\%, \%*\%, \%\%, <-, <<-, /},% other keywords 
        deletekeywords={c}% remove keywords 
} 
\Newassociation{solexo}{Solution}{ans}

\usetikzlibrary{automata}
\usetikzlibrary{patterns}


%\usepackage{courier}
%\usepackage{algpseudocode}
%\usepackage[comma]{natbib}
\usepackage{algorithm}
\usepackage{algorithmicx}
\usepackage{algpseudocode}

\usepackage{pdfpages}

\usepackage{minitoc}

\usepackage{makeidx}

%\usepackage{index} %%%% pour faire deux index
%\newindex{models}{otx}{otd}{List of models}  %%%% pour faire un index des modeles étudiés
                                                   %%%% entrer les cles d'index sous la forme \index[models]{le-modele}
%%% \newindex{not}{notx}{notd}{List of notation}  %%%% pour l'index des notations; inutile
%%%% makeindex mainMC-tome1.otx -t mainMC-tome1.otg -s svind.ist -o mainMC-tome1.otd %%% pour compiler en ligne de commande

\makeindex             % used for the subject index
                       % please use the style svind.ist with your makeindex program


%%%%%%%%%%%%%%%%%%%%%%%%%%%  COMMENT THESE OUT AT FINAL COMPLIE %%%%%%%%%%%%%%%%%%%%%%%%%%%%%%%%%%%%%
%% pdfsync manual says to comment it out at final version because it can change the page breaks
\usepackage[novbox]{pdfsync}           %  [novbox] helps
%\usepackage[notref,notcite]{showkeys}  %  comment out for final version
%\renewcommand*\showkeyslabelformat[1]{\fbox{\normalfont\scriptsize\sffamily#1}}   % comment for the final version
%\usepackage{showidx}                   % use to check the index - comment out for final version  %%%%%%%%%

% [disable] shuts it off:
%\usepackage[disable]{todonotes}
%%%%%%%%%%%%%%%%%%%%%%%%%%%%%%%%%%%%%%%%%%%%%%%%%%%%%%%%%%%%%%%%%%%%%%%%%%%%%%%%%%%%%%%%%%%%%%%%%%%%%

\renewcommand\qedsymbol{\hfill \ensuremath{\Box}}
\spnewtheorem*{proof_duplicate}{Proof}{\itshape}{\rmfamily}
\renewenvironment{proof}{\begin{proof_duplicate}}{\qedsymbol\end{proof_duplicate}}
\renewcommandx\prob[2][1=,2=]{\ensuremath{{\mathbb P}_{#1}^{#2}}}


\definecolor{dblue}{rgb}{0,.1,.6}  % used for dark blue hyperrefs
\definecolor{dblue2}{RGB}{0,75,90}  % used for R output in listings

%\usepackage{hyperref}
\usepackage[hyperindex=true]{hyperref}

\usepackage{aliascnt}
\usepackage{cleveref}




\newtheorem{kwd}{Keywords}[chapter]
\newaliascnt{lemma}{theorem}
\renewtheorem{lemma}[lemma]{Lemma}
\aliascntresetthe{lemma}
\crefname{lemma}{Lemma}{lemmas}
\Crefname{Lemma}{Lemma}{Lemmas}

\renewtheorem{claim}{Claim}[chapter]
\crefname{claim}{Claim}{Claims}
\Crefname{claim}{Claim}{Claims}


\newaliascnt{proposition}{theorem}
\renewtheorem{proposition}[proposition]{Proposition}
\aliascntresetthe{proposition}
\crefname{proposition}{Proposition}{Propositions}
\Crefname{Proposition}{Proposition}{Propositions}

\newaliascnt{corollary}{theorem}
\renewtheorem{corollary}[corollary]{Corollary}
\aliascntresetthe{corollary}
\crefname{corollary}{corollary}{corollaries}
\Crefname{Corollary}{Corollary}{Corollaries}

%\newtheoremstyle{styledef}
%  {6pt}% space above
%  {6pt}% space below
%  {\sffamily}%body font
%  {0em}%indent amount
%  {\bfseries}%Theorem head
%  {}%punctuation
%  {1.5em}%space after theorem head
%  {}
%
%\theoremstyle{styledef}
\newaliascnt{definition}{theorem}
%\renewtheorem{definition}[definition]{Definition}
\aliascntresetthe{definition}
\crefname{definition}{definition}{definitions}
\Crefname{Definition}{Definition}{Definitions}

\theoremstyle{exercise}

\newaliascnt{remark}{theorem}
\renewtheorem{remark}[remark]{Remark}
\aliascntresetthe{remark}
\crefname{remark}{remark}{remarks}
\Crefname{remark}{Remark}{Remarks}
% \renewenvironment{remark}{\stepcounter{theorem} \noindent {\bf Remark
% \arabic{chapter}.\arabic{theorem}}\begin{rm} }{\end{rm}\hfill $\triangleleft$}

\newaliascnt{example}{theorem}
\renewtheorem{example}[example]{Example}
\aliascntresetthe{example}
\crefname{example}{example}{examples}
\Crefname{example}{Example}{Examples}



\newtheorem{assumption}[theorem]{\textbf{H}}
\Crefname{assumption}{\textbf{H}}{\textbf{H}}
\crefname{assumption}{\textbf{H}}{\textbf{H}}

\newaliascnt{xmpl}{theorem}
\spnewtheorem{xmpl}[xmpl]{Example}{\bf}{}
\aliascntresetthe{xmpl}
\crefname{xmpl}{example}{examples}
\Crefname{xmpl}{Example}{Examples}



\newaliascnt{rmrk}{theorem}
\spnewtheorem{rmrk}[rmrk]{Remark}{\bf}{}
\aliascntresetthe{rmrk}
\crefname{rmrk}{remark}{remarks}
\Crefname{rmrk}{Remark}{Remarks}

%\newcommand\enxmpl{\hfill\ensuremath{\PHrosette}\end{xmpl}}
\newcommand\enxmpl{\hfill\ensuremath{\blacktriangleleft}\end{xmpl}}
\newcommand\enrmrk{\hfill\ensuremath{\triangle}\end{remark}}
\newcommand\begrmrkk{\begin{rmrk}}
\newcommand\enrmrkkk{\hfill\ensuremath{\blacktriangle}\end{rmrk}}

% \newcommand\boxendxmpl{\hfill \ensuremath{\blacksquare}}
% \spnewtheorem{xmpl_duplicate}[theorem]{Example}{\bf}{\rmfamily}
% \renewenvironment{xmpl}{\begin{xmpl_duplicate}}{\boxendxmpl\end{xmpl_duplicate}}

\crefname{theorem}{Theorem}{Theorems}
\Crefname{theorem}{Theorem}{Theorems}
\crefname{corollary}{Corollary}{Corollaries}
\Crefname{corollary}{Corollary}{Corollaries}
\Crefname{chapter}{Chapter}{Chapters}
\crefname{chapter}{Chapter}{Chapters}


\setcounter{tocdepth}{1}

% choisir l une des deux lignes ci dessous suivant qu on veuille voir les commentaires ou non
\usepackage[textsize=footnotesize]{todonotes} % pour voir les commentaires
%\usepackage[disable]{todonotes} % pour ne pas voir les commentaires.

% \usepackage[textwidth=3cm, textsize=footnotesize]{todonotes}
% \setlength{\marginparwidth}{3cm} % this goes with todonotes




\theoremseparator{.}
\theorembodyfont{\upshape}

\newcounter{exo}
\newaliascnt{exercise}{exo}
\renewtheorem{exercise}{Exercise}[chapter]
 \crefname{exercise}{Exercise}{Exercises}
 \Crefname{exercise}{Exercise}{Exercises}

\renewenvironment{proof}%
 {\noindent \small {\textsc{Proof}.} }%
 { \hfill$\blacksquare$\\}               % Not needed if using AMSART style






