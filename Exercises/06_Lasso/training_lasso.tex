\documentclass[a4paper,10pt,fleqn]{article}

\usepackage{a4wide,amsmath,amsthm,amssymb,bbm,fancyhdr}
\usepackage{ifthen,color,enumerate,comment,dsfont,pdfsync,framed,todonotes,enumitem}

\newcommand{\titre}[1]{\textbf{\textsc{#1}}}

\RequirePackage[T1]{fontenc}

\usepackage[latin1]{inputenc}
\usepackage{graphicx}
\usepackage{dsfont}
\usepackage{enumitem,url,hyperref}
\newcommand{\eqsp}{\,}
\newcommand{\R}{\ensuremath{\mathbb{R}}}
\newcommand{\calF}{\mathcal{F}}
\newcommand{\rmd}{\mathrm{d}}
\newcommand{\N}{\mathbb{N}}
\newcommand{\rset}{\ensuremath{\mathbb{R}}}
\renewcommand{\P}{\ensuremath{\operatorname{P}}}
\newcommand{\bP}{\mathbb{P}}
\newcommand{\E}{\ensuremath{\mathbb{E}}}
\newcommand{\rme}{\ensuremath{\mathrm{e}}}
\newcommand{\calH}{\ensuremath{\mathcal{H}}}
\newcommand{\xset}{\ensuremath{\mathsf{X}}}
\newcommand{\V}{\ensuremath{\mathbb{V}}}
\newcommand{\Sb}{\ensuremath{\mathbb{S}}}
\newcommand{\gaus}{\ensuremath{\mathcal{N}}}
\newcommand{\HH}{\ensuremath{\mathcal{H}}}
\newcommand{\F}{\ensuremath{\mathcal{F}}}
\newcommand{\W}{\ensuremath{\mathcal{W}}}
\newcommand{\X}{\ensuremath{\mathcal{X}}}
\newcommand{\1}{\ensuremath{\mathbbm{1}}}
\newcommand{\dlim}{\ensuremath{\stackrel{\mathcal{L}}{\longrightarrow}}}
\newcommand{\plim}{\ensuremath{\stackrel{\mathrm{P}}{\longrightarrow}}}
\newcommand{\PP}{\ensuremath{\mathbb{P}}}
\newcommand{\p}{\ensuremath{\mathbb{P}}}
\newcommand{\eps}{\varepsilon}
\newcommand{\bE}{\mathbb{E}}

\newcommand{\pa}[1]{\left(#1\right)}
\newcommand{\hatk}{\widehat K}
\newcommand{\f}{\varphi}
\newcommand{\Id}{\textsf{Id}}
\newcommand{\bfU}{\mathbf{U}}
\newcommand{\bfX}{\mathbf{X}}
\newcommand{\bfs}{\mathbf{\Sigma}}
\newcommand{\bfA}{\mathbf{A}}
\newcommand{\bfV}{\mathbf{V}}
\newcommand{\bfB}{\mathbf{B}}
\newcommand{\bfI}{\mathbf{I}}
\newcommand{\bfD}{\mathbf{D}}
\newcommand{\bfK}{\mathbf{K}}
\newcommand{\argmin}{\mathop{\textrm{argmin}}}
\newcommand{\argmax}{\mathop{\textrm{argmax}}}
\newcommand{\crit}{\mathop{\textrm{crit}}}
\newcommand{\C}{\mathcal{C}}
\newcommand{\pc}{\pi_{\mathcal{C}}}
\newcommand{\param}{\theta}


\begin{document}

\noindent\hrulefill

\begin{center}
\textsc{Penalizations $\mathrm{L}^1$ and $\mathrm{L}^2$}
\end{center}
\hrulefill

\medskip

\section{Warm-up}
Consider a model given by 
$$
Y = X\theta_* + \varepsilon\,,
$$
where $X\in\mathbb{R}^{n\times d}$ and $\varepsilon \sim \mathcal{N}(0,\sigma_*^2I_n)$. The Ridge estimator  is defined for all $\lambda>0$ by:
$$
\widehat{\theta}_{\lambda}\in\mathrm{Argmin}_{\theta\in\mathbb{R}^d} \mathcal{L}(\theta)\quad\mathrm{with} \quad \mathcal{L}(\theta) = \|Y-X\theta\|_2^2 + \lambda \|\theta\|_2^2\,.
 $$
For all $\lambda>0$, the excess risk is given by
\begin{multline*}
\bE\left[\mathsf{R}(\widehat \param^{\mathrm{ridge}}_{n,\lambda}) - \mathsf{R}(\param_\star)\right] = \lambda^2\param_\star^\top\left(\frac{1}{n}X^\top X + \lambda I_d\right)^{-2} \frac{1}{n}X^\top X \param_\star \\
+\frac{\sigma_\star^{2}}{n}\mathrm{Trace}\left((n^{-1}X^\top X)^2(n^{-1}X^\top X + \lambda I_d)^{-2}\right)\eqsp.
\end{multline*}

\begin{enumerate}
\item Prove that 
$$
\bE\left[\mathsf{R}(\widehat \param^{\mathrm{ridge}}_n) - \mathsf{R}(\param_\star)\right]  \leqslant \frac{\lambda}{2}\|\param_\star\|_2^2 + \frac{\sigma_\star^{2}}{2n\lambda}\mathrm{Trace}\left(n^{-1}X^\top X\right)\eqsp.
$$

\vspace{.2cm}

{\em

}

\item Propose an "optimal" value for $\lambda$ and  compute the associated excess risk.

\vspace{.2cm}

{\em

}
\end{enumerate}

\section{Elastic-Net}
Consider a model given by 
$$
Y = X\theta_* + \varepsilon\,,
$$
where $X\in\mathbb{R}^{n\times d}$ and $\varepsilon \sim \mathcal{N}(0,\sigma_*^2I_n)$. The Elastic-Net estimator involves both $\mathrm{L}^1$ and $\mathrm{L}^2$ penalties. It is defined for all $\lambda,\mu>0$ by:
$$
\widehat{\theta}_{\lambda,\mu}\in\mathrm{Argmin}_{\theta\in\mathbb{R}^d} \mathcal{L}(\theta)\quad\mathrm{with} \quad \mathcal{L}(\theta) = \|Y-X\theta\|_2^2 + \lambda \|\theta\|_2^2 + \mu \|\theta\|_1\,.
 $$
In the following, we assume that for all $1\leq j\leq d$, the $j$-th column of $X$ satisfies $\|\mathbf{X}_j\|_2 = 1$.
\begin{enumerate}
\item For all $1\leq j \leq d$ provide the partial derivative of $\mathcal{L}$ with respect to $\theta_j$ for $\theta_j\neq 0$. 
\item Provide an expression of the answer of the first question with $R_j = \mathbf{X}_j^\top(Y - \sum_{k\neq j}\theta_k\mathbf{X}_k)$.
\item Assume that $\theta_k$, $1\leq k\neq j \leq d$ are fixed and assume that the minimum of $\theta_j \mapsto \mathcal{L}(\theta)$ is reached at a $\theta_j\neq 0$. Prove that the sign of $\theta_j$ is the same as the signe of $R_j$ and conclude.
\item Provide an algorithm to obtain an approximation of $\widehat{\theta}_{\lambda,\mu}$.
\end{enumerate}


\end{document}





	
