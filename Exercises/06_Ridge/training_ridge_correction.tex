\documentclass[a4paper,10pt,fleqn]{article}

\usepackage{a4wide,amsmath,amsthm,amssymb,bbm,fancyhdr}
\usepackage{ifthen,color,enumerate,comment,dsfont,pdfsync,framed,todonotes,enumitem}

\newcommand{\titre}[1]{\textbf{\textsc{#1}}}

\RequirePackage[T1]{fontenc}

\usepackage[latin1]{inputenc}
\usepackage{graphicx}
\usepackage{dsfont}
\usepackage{enumitem,url,hyperref}
\newcommand{\eqsp}{\,}
\newcommand{\R}{\ensuremath{\mathbb{R}}}
\newcommand{\calF}{\mathcal{F}}
\newcommand{\rmd}{\mathrm{d}}
\newcommand{\N}{\mathbb{N}}
\newcommand{\rset}{\ensuremath{\mathbb{R}}}
\renewcommand{\P}{\ensuremath{\operatorname{P}}}
\newcommand{\bP}{\mathbb{P}}
\newcommand{\E}{\ensuremath{\mathbb{E}}}
\newcommand{\rme}{\ensuremath{\mathrm{e}}}
\newcommand{\calH}{\ensuremath{\mathcal{H}}}
\newcommand{\xset}{\ensuremath{\mathsf{X}}}
\newcommand{\V}{\ensuremath{\mathbb{V}}}
\newcommand{\Sb}{\ensuremath{\mathbb{S}}}
\newcommand{\gaus}{\ensuremath{\mathcal{N}}}
\newcommand{\HH}{\ensuremath{\mathcal{H}}}
\newcommand{\F}{\ensuremath{\mathcal{F}}}
\newcommand{\W}{\ensuremath{\mathcal{W}}}
\newcommand{\X}{\ensuremath{\mathcal{X}}}
\newcommand{\1}{\ensuremath{\mathbbm{1}}}
\newcommand{\dlim}{\ensuremath{\stackrel{\mathcal{L}}{\longrightarrow}}}
\newcommand{\plim}{\ensuremath{\stackrel{\mathrm{P}}{\longrightarrow}}}
\newcommand{\PP}{\ensuremath{\mathbb{P}}}
\newcommand{\p}{\ensuremath{\mathbb{P}}}
\newcommand{\eps}{\varepsilon}
\newcommand{\bE}{\mathbb{E}}

\newcommand{\pa}[1]{\left(#1\right)}
\newcommand{\hatk}{\widehat K}
\newcommand{\f}{\varphi}
\newcommand{\Id}{\textsf{Id}}
\newcommand{\bfU}{\mathbf{U}}
\newcommand{\bfX}{\mathbf{X}}
\newcommand{\bfs}{\mathbf{\Sigma}}
\newcommand{\bfA}{\mathbf{A}}
\newcommand{\bfV}{\mathbf{V}}
\newcommand{\bfB}{\mathbf{B}}
\newcommand{\bfI}{\mathbf{I}}
\newcommand{\bfD}{\mathbf{D}}
\newcommand{\bfK}{\mathbf{K}}
\newcommand{\argmin}{\mathop{\textrm{argmin}}}
\newcommand{\argmax}{\mathop{\textrm{argmax}}}
\newcommand{\crit}{\mathop{\textrm{crit}}}
\newcommand{\C}{\mathcal{C}}
\newcommand{\pc}{\pi_{\mathcal{C}}}
\newcommand{\param}{\theta}

% Style


\begin{document}

%\noindent EMINES \hfill \\%ISUP - Sorbonne Universit\'e \\
 %2022-2023

\noindent\hrulefill

\begin{center}
\textsc{Full rank linear regression}
\end{center}
\hrulefill

\medskip

\section{Warm-up}
Let $X$ be a random vector in $\rset^d$ with mean $\mu\in\rset^d$ and covariance matrix $\Sigma\in\rset^{d\times d}$ and $A$ a symmetric matrix in $\rset^{d\times d}$. Then,
$$
\bE[X^\top A X] = \mu^\top A \mu + \mathrm{Trace}(A\Sigma)\eqsp.
$$

\vspace{.2cm}

{\em
\noindent As $X^\top A X$ is a real number, $\bE[X^\top A X] = \bE[\mathrm{Trace}(X^\top A X)] =  \bE[\mathrm{Trace}(AXX^\top )]$. By linearity, $\bE[X^\top A X] = \mathrm{Trace}(A\bE[XX^\top])$ which yields,
$$
\bE[X^\top A X] =   \mathrm{Trace}(A\{\mathbb{V}[X] + \bE[X]\bE[X]^\top\}) =  \mu^\top A \mu + \mathrm{Trace}(A\Sigma)\eqsp.
$$
}


\section{Student's t-statistics}
We assume that  for all $1\leqslant i \leqslant n$, $Y_i = X^\top_i \param_{\star} + \varepsilon_i$ for some unknown $\param_\star\in\rset^d$ where the $(\varepsilon_i)_{1\leqslant i\leqslant n}$ are i.i.d. random variables with distribution $\mathcal{N}(0,\sigma_*^2)$. Let  $\varepsilon\in\rset^n$  be the random vector such that  for all $1\leqslant i \leqslant n$, the $i$-th component of $\varepsilon$  is $\varepsilon_i$. The model is then written $Y = X \param_{\star} + \varepsilon$. Assume that $X$ has full rank and that $\widehat \theta_n = (X^\top X)^{-1}X^\top Y$ and $\widehat \sigma^2_n = \|Y-X\widehat \theta_n\|^2/(n-d)$.
\begin{enumerate}
\item For all $1\leqslant j \leqslant d$, show that
$$
\frac{\widehat \param_{n,j} -\param_{\star,j}}{\widehat\sigma_{n}\sqrt{(X^TX)_{j,j}^{-1}}} \sim \mathcal{S}(n-d)\eqsp,
$$
where $\mathcal{S}(n-d)$ is the Student's t-distribution with $n-d$ degrees of freedom, i.e. the law of $X/\sqrt{Y/(n-d)}$ where $X\sim\mathcal{N}(0,1)$ is independent of $Y\sim\chi^2(n-d)$.

\vspace{.2cm}

{\em
By definition, for all $1\leqslant j \leqslant d$,  
\[
\frac{\widehat \param_{n,j} -\param_{\star,j}}{\widehat\sigma_{n}\sqrt{(X^\top X)_{j,j}^{-1}}} = \frac{\sigma_\star^{-1}(\widehat \param_{n,j} -\param_{\star,j})}{\sigma_\star^{-1}\widehat\sigma_{n}\sqrt{(X^\top X)_{j,j}^{-1}}} = \frac{e^\top_j(\sigma_\star^{-1}(\widehat \param_{n} -\param_{\star}))}{\sigma_\star^{-1}\widehat\sigma_{n}\sqrt{(X^\top X)_{j,j}^{-1}}}\eqsp.
\]
Note that $\sigma_\star^{-1}(\widehat \param_{n} -\param_{\star}) \sim \mathcal{N}(0,(X^\top X)^{-1})$ so that $e^\top_j(\sigma_\star^{-1}(\widehat \param_{n} -\param_{\star}))\sim \mathcal{N}(0,e^\top_j(X^\top X)^{-1}e_j)$ and 
\[
\frac{e^\top_j(\sigma_\star^{-1}(\widehat \param_{n} -\param_{\star}))}{\sqrt{(X^\top X)_{j,j}^{-1}}}\sim\mathcal{N}(0,1)\eqsp.
\]
In addition,
\[
\sigma_\star^{-1}\widehat\sigma_{n} = \sqrt{\sigma_\star^{-2}\widehat\sigma^2_{n}} = \sqrt{\|\sigma_\star^{-1}(I_n - X(X^\top X)^{-1}X^\top)\varepsilon\|_2^2/(n-d)}\eqsp,
\]
where $\sigma_\star^{-2}\widehat\sigma^2_{n} = \|\sigma_\star^{-1}(I_n - X(X^\top X)^{-1}X^\top)\varepsilon\|_2^2 \sim \chi^2(n-d)$. The proof is concluded by noting that $\widehat \param_n$ and  $\widehat\sigma^2_{n}$ are independent.
}
\item Provide a confidence interval with confidence level $1-\alpha$ for $\param_{\star,j}$.

\vspace{.2cm}

{\em
For $\alpha\in(0,1)$, if $s_{1-\alpha/2}^{n-d}$ denotes the quantile of order $1-\alpha/2$ of the law $\mathcal{S}(n-d)$, then 
\[
\bP\left(\left|\frac{\widehat \param_{n,j} -\param_{\star,j}}{\widehat\sigma_{n}\sqrt{(X^TX)_{j,j}^{-1}}}\right| \leqslant s_{1-\alpha/2}^{n-d}\right) = 1-\alpha\eqsp.
\]
Therefore, 
\[
I^{n-p}_{n,j}(\param_{\star}) = \left[\widehat \param_{n,j} -\widehat\sigma_{n}s_{1-\alpha/2}^{n-d}\sqrt{(X^\top X)_{j,j}^{-1}}\eqsp;\eqsp \widehat \param_{n,j} + \widehat\sigma_{n}s_{1-\alpha/2}^{n-d}\sqrt{(X^\top X)_{j,j}^{-1}}\right]
\]
 is a confidence interval for $\param_{\star,j}$ with confidence level $1-\alpha$. 
}
\end{enumerate}
\section{Random design}
Consider the regression model given by
$$
Y = X\param_{\star}+ \varepsilon\eqsp,
$$
where $X\in\rset^{n\times d}$ the $(\varepsilon_{i})_{1\leqslant i \leqslant n}$ are i.i.d. centered Gaussian random variables with variance $\sigma_{\star}^2$ and independent of $(X_{i})_{1\leqslant i \leqslant n}$ which are assumed to be random. Assume that $X^\top X$ has full rank and that $\param_\star$ is estimated by 
$$
\widehat \param_n = (X^\top X)^{-1}X^\top Y\eqsp.
$$
\begin{enumerate}
\item  Compute the excess risk $\mathsf{R}(\param)-\mathsf{R}(\param_\star)$, where $\mathsf{R}(\param) = n^{-1}\bE[\|Y - X^\top \param\|_2^2]$.

\vspace{.2cm}

{\em
By definition, using that $\bE[\varepsilon]=0$,
\begin{align*}
\mathsf{R}(\param) = n^{-1}\bE[\|Y - X \param\|_2^2] = n^{-1}\mathsf{R}(\param) &= \bE[\|X\param_\star + \varepsilon - X\param\|_2^2]\eqsp,\\
&= n^{-1}\bE[\|X \param_\star - X\param\|_2^2] + n^{-1}\bE[\|\varepsilon\|_2^2]\eqsp,\\
&=(\param_\star - \param)^\top n^{-1}\bE[X^\top X](\param_\star - \param) + \sigma_\star^2\eqsp.
\end{align*}
Therefore, $\mathsf{R}(\param)-\mathsf{R}(\param_\star) = (\param_\star - \param)^\top n^{-1}\bE[X^\top X](\param_\star - \param) $.
}
\item  Compute then the excess risk $\bE[\mathsf{R}(\widehat \param_n)-\mathsf{R}(\param_\star)]$.

\vspace{.2cm}

{\em
By the previous question,
$$
\bE[\mathsf{R}(\widehat \param_n)-\mathsf{R}(\param_\star)] = n^{-1}\bE[(\param_\star - \widehat \param_n)^\top \bE[X^\top X](\param_\star - \widehat \param_n) ]\eqsp.
$$
Since $\widehat \param_n$ is an unbiased estimate of $\param_\star$,
\begin{align*}
\bE[\mathsf{R}(\widehat \param_n)-\mathsf{R}(\param_\star)] &= n^{-1}\bE[(\param_\star - \bE[\widehat \param_n])^\top \bE[X^\top X](\param_\star - \bE[\widehat \param_n]) ]\eqsp,\\
&= n^{-1}\mathrm{Trace}\left(\bE[X^\top X]\mathbb{V}[\widehat \param_n]\right)\eqsp,\\
&=\frac{\sigma_\star}{n}\mathrm{Trace}\left(\bE[X^\top X]\bE\left[(X^\top X)^{-1}\right]\right)\eqsp.
\end{align*}
}

\end{enumerate}

\section{Fisher statistics (bonus)}
Consider the regression model given by
$$
Y = X\param_{\star}+ \varepsilon\eqsp,
$$
where $X\in\rset^{n\times d}$ and the $(\varepsilon_{i})_{1\leqslant i \leqslant n}$ are i.i.d. centered Gaussian random variables with variance $\sigma_{\star}^2$. Assume that $X^\top X$ has full rank and that $\param_\star$ and $\sigma_{\star}^2$ are estimated by 
$$
\widehat \param_n = (X^\top X)^{-1}X^\top Y\quad\mathrm{and}\quad \widehat \sigma^2_n =\frac{\|Y - X\widehat \param_n \|^2}{n-d}\eqsp.
$$
\begin{enumerate}
\item  Let $L$ be a $\rset^{q\times d}$ matrix with rank $q\leqslant d$. Show that
$$
\frac{(\widehat \param_{n} -\param_{\star})^\top L^\top(L(X^\top X)^{-1}L^\top)^{-1}L(\widehat \param_{n} -\param_{\star})}{q\widehat\sigma^2_{n}} \sim \mathcal{F}(q,n-d)\eqsp,
$$
where $\mathcal{F}(q,n-d)$ is the Fisher distribution with $q$ and $n-d$ degrees of freedom, i.e. the law of $(X/q)/(Y/(n-d))$ where $X\sim\chi^2(q)$ is independent of $Y\sim\chi^2(n-d)$.

\vspace{.2cm}

{\em
Note that $\mathrm{rank}(L(X^\top X)^{-1}L^\top) = \mathrm{rank}(LL^\top)  = q$. The matrix $L(X^\top X)^{-1}L^\top$ is therefore positive definite. There exists a diagonal matrix $D\in\rset^{q\times q}$ with positive diagonal terms and an orthogonal matrix $Q\in\rset^{q\times q}$ such that $L(X^\top X)^{-1}L^\top = QDQ^{-1}$. The matrix $(L(X^\top X)^{-1}L^\top)^{-1/2}$ may be defined as $(L(X^\top X)^{-1}L^\top)^{-1/2} = QD^{-1/2}Q^{-1}$. 

It is then enough to note that $(L(X^\top X)^{-1}L^\top)^{-1/2}L(\widehat \param_{n} -\param_{\star})/\sigma_{\star}\sim \mathcal{N}(0,I_q)$. Therefore,
\begin{multline*}
\sigma^{-2}_{\star}\|(L(X^\top X)^{-1}L^\top)^{-1/2}L(\widehat \param_{n} -\param_{\star})\|^2 \\
= (\widehat \param_{n} -\param_{\star})^\top L^\top(L(X^\top X)^{-1}L^\top)^{-1}L(\widehat \param_{n} -\param_{\star})/\sigma^2_{\star} \sim \chi^2(q)\eqsp.
\end{multline*}
On the other hand,we know that 
$$
(n-d)\sigma^{-2}_{\star}\widehat\sigma^2_{n} \sim \chi^2(n-d)\eqsp.
$$
The proof is concluded by noting that $\widehat \param_n$ and  $\widehat\sigma^2_{n}$ are independent.
}
\item Using the previous question, build a confidence region with confidence level $1-\alpha\in(0,1)$ for $\param_\star$.

\vspace{.2cm}

{\em
By the previous question, for $\alpha\in(0,1)$, if $f_{1-\alpha}^{q,n-d}$ denotes the quantile of order $1-\alpha$ of the law $\mathcal{F}(q,n-p)$, then 
\[
\bP\left(\param_{\star}\in \left\{\param\in\rset^d\eqsp;\eqsp (\widehat \param_{n} -\param)^\top L^\top(L(X^\top X)^{-1}L^\top)^{-1}L(\widehat \param_{n} -\param)\leqslant q\widehat\sigma^2_{n}f_{1-\alpha}^{q,n-d}\right\}\right) = 1-\alpha\eqsp.
\]
Therefore, 
\[
I^{q,n-d}_{n}(\param_{\star}) = \left\{\param\in\rset^d\eqsp;\eqsp (\widehat \param_{n} -\param)^\top L^\top(L(X^\top X)^{-1}L^\top)^{-1}L(\widehat \param_{n} -\param)\leqslant q\widehat\sigma^2_{n}f_{1-\alpha}^{q,n-d}\right\}
\]
 is a confidence region for $\param_{\star}$ with confidence level $1-\alpha$. 
}
\end{enumerate}



%\section{Multivariate linear regression in practice}
%
%See Python Notebook on Moodle or \url{https://sylvainlc.github.io/}.


\end{document}





	
