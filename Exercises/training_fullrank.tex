\documentclass[a4paper,10pt,fleqn]{article}

\usepackage{a4wide,amsmath,amsthm,amssymb,bbm,fancyhdr}
\usepackage{ifthen,color,enumerate,comment,dsfont,pdfsync,framed,todonotes,enumitem}

\newcommand{\titre}[1]{\textbf{\textsc{#1}}}

\RequirePackage[T1]{fontenc}

\usepackage[latin1]{inputenc}
\usepackage{graphicx}
\usepackage{dsfont}
\usepackage{enumitem,url,hyperref}
\newcommand{\eqsp}{\,}
\newcommand{\R}{\ensuremath{\mathbb{R}}}
\newcommand{\calF}{\mathcal{F}}
\newcommand{\rmd}{\mathrm{d}}
\newcommand{\N}{\mathbb{N}}
\newcommand{\rset}{\ensuremath{\mathbb{R}}}
\renewcommand{\P}{\ensuremath{\operatorname{P}}}
\newcommand{\bP}{\mathbb{P}}
\newcommand{\E}{\ensuremath{\mathbb{E}}}
\newcommand{\rme}{\ensuremath{\mathrm{e}}}
\newcommand{\calH}{\ensuremath{\mathcal{H}}}
\newcommand{\xset}{\ensuremath{\mathsf{X}}}
\newcommand{\V}{\ensuremath{\mathbb{V}}}
\newcommand{\Sb}{\ensuremath{\mathbb{S}}}
\newcommand{\gaus}{\ensuremath{\mathcal{N}}}
\newcommand{\HH}{\ensuremath{\mathcal{H}}}
\newcommand{\F}{\ensuremath{\mathcal{F}}}
\newcommand{\W}{\ensuremath{\mathcal{W}}}
\newcommand{\X}{\ensuremath{\mathcal{X}}}
\newcommand{\1}{\ensuremath{\mathbbm{1}}}
\newcommand{\dlim}{\ensuremath{\stackrel{\mathcal{L}}{\longrightarrow}}}
\newcommand{\plim}{\ensuremath{\stackrel{\mathrm{P}}{\longrightarrow}}}
\newcommand{\PP}{\ensuremath{\mathbb{P}}}
\newcommand{\p}{\ensuremath{\mathbb{P}}}
\newcommand{\eps}{\varepsilon}
\newcommand{\bE}{\mathbb{E}}

\newcommand{\pa}[1]{\left(#1\right)}
\newcommand{\hatk}{\widehat K}
\newcommand{\f}{\varphi}
\newcommand{\Id}{\textsf{Id}}
\newcommand{\bfU}{\mathbf{U}}
\newcommand{\bfX}{\mathbf{X}}
\newcommand{\bfs}{\mathbf{\Sigma}}
\newcommand{\bfA}{\mathbf{A}}
\newcommand{\bfV}{\mathbf{V}}
\newcommand{\bfB}{\mathbf{B}}
\newcommand{\bfI}{\mathbf{I}}
\newcommand{\bfD}{\mathbf{D}}
\newcommand{\bfK}{\mathbf{K}}
\newcommand{\argmin}{\mathop{\textrm{argmin}}}
\newcommand{\argmax}{\mathop{\textrm{argmax}}}
\newcommand{\crit}{\mathop{\textrm{crit}}}
\newcommand{\C}{\mathcal{C}}
\newcommand{\pc}{\pi_{\mathcal{C}}}
\newcommand{\param}{\theta}

% Style


\begin{document}

\noindent Apprentissage statistique \hfill ISUP - Sorbonne Universit\'e \\
 2022-2023

\noindent\hrulefill

\begin{center}
\textsc{Full rank linear regression}
\end{center}
\hrulefill

\medskip


\section{Fisher statistics}
Consider the regression model given by
$$
Y = X\param_{\star}+ \varepsilon\eqsp,
$$
where $X\in\rset^{n\times d}$ and the $(\varepsilon_{i})_{1\leqslant i \leqslant n}$ are i.i.d. centered Gaussian random variables with variance $\sigma_{\star}^2$. Assume that $X^\top X$ has full rank and that $\param_\star$ and $\sigma_{\star}^2$ are estimated by 
$$
\widehat \param_n = (X^\top X)^{-1}X^\top Y\quad\mathrm{and}\quad \widehat \sigma^2_n =\frac{\|Y - X\widehat \param_n \|^2}{n-d}\eqsp.
$$
\begin{enumerate}
\item  Let $L$ be a $\rset^{q\times d}$ matrix with rank $q\leqslant d$. Show that
$$
\frac{(\widehat \param_{n} -\param_{\star})^\top L^\top(L(X^\top X)^{-1}L^\top)^{-1}L(\widehat \param_{n} -\param_{\star})}{q\widehat\sigma^2_{n}} \sim \mathcal{F}(q,n-d)\eqsp,
$$
where $\mathcal{F}(q,n-d)$ is the Fisher distribution with $q$ and $n-d$ degrees of freedom, i.e. the law of $(X/q)/(Y/(n-d))$ where $X\sim\chi^2(q)$ is independent of $Y\sim\chi^2(n-d)$.
\item Using the previous question, build a confidence region with confidence level $1-\alpha\in(0,1)$ for $\param_\star$.
\end{enumerate}

\section{Random design}
Consider the regression model given by
$$
Y = X\param_{\star}+ \varepsilon\eqsp,
$$
where $X\in\rset^{n\times d}$ the $(\varepsilon_{i})_{1\leqslant i \leqslant n}$ are i.i.d. centered Gaussian random variables with variance $\sigma_{\star}^2$ and independent of $(X_{i})_{1\leqslant i \leqslant n}$ which are assumed to be random. Assume that $X^\top X$ has full rank and that $\param_\star$ is estimated by 
$$
\widehat \param_n = (X^\top X)^{-1}X^\top Y\eqsp.
$$
\begin{enumerate}
\item  Compute the excess risk $\mathsf{R}(\param)-\mathsf{R}(\param_\star)$, where $\mathsf{R}(\param) = n^{-1}\bE[(Y - X^\top \param)^2]$.

%\vspace{.2cm}
%
%{\em
%By definition, using that $\bE[\varepsilon]=0$,
%\begin{align*}
%\mathsf{R}(\param) = n^{-1}\bE[\|Y - X \param\|_2^2] = n^{-1}\mathsf{R}(\param) &= \bE[\|X\param_\star + \varepsilon - X\param\|_2^2]\eqsp,\\
%&= n^{-1}\bE[\|X \param_\star - X\param\|_2^2] + n^{-1}\bE[\|\varepsilon\|_2^2]\eqsp,\\
%&=(\param_\star - \param)^\top n^{-1}\bE[X^\top X](\param_\star - \param) + \sigma_\star^2\eqsp.
%\end{align*}
%Therefore, $\mathsf{R}(\param)-\mathsf{R}(\param_\star) = (\param_\star - \param)^\top n^{-1}\bE[X^\top X](\param_\star - \param) $.
%}
\item  Compute then the excess risk $\bE[\mathsf{R}(\widehat \param_n)-\mathsf{R}(\param_\star)]$.

%\vspace{.2cm}
%
%{\em
%By the previous question,
%$$
%\bE[\mathsf{R}(\widehat \param_n)-\mathsf{R}(\param_\star)] = n^{-1}\bE[(\param_\star - \widehat \param_n)^\top \bE[X^\top X](\param_\star - \widehat \param_n) ]\eqsp.
%$$
%Since $\widehat \param_n$ is an unbiased estimate of $\param_\star$,
%\begin{align*}
%\bE[\mathsf{R}(\widehat \param_n)-\mathsf{R}(\param_\star)] &= n^{-1}\bE[(\param_\star - \bE[\widehat \param_n])^\top \bE[X^\top X](\param_\star - \bE[\widehat \param_n]) ]\eqsp,\\
%&= n^{-1}\mathrm{Trace}\left(\bE[X^\top X]\mathbb{V}[\widehat \param_n]\right)\eqsp,\\
%&=\frac{\sigma_\star}{n}\mathrm{Trace}\left(\bE[X^\top X]\bE\left[(X^\top X)^{-1}\right]\right)\eqsp.
%\end{align*}
%}

\end{enumerate}

\section{Multivariate linear regression in practice}

See Python Notebook on Moodle or \url{https://sylvainlc.github.io/}.


\end{document}





	
